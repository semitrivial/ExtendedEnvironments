\documentclass{article}

% if you need to pass options to natbib, use, e.g.:
%     \PassOptionsToPackage{numbers, compress}{natbib}
% before loading neurips_2021

% ready for submission
\usepackage{neurips_2021}

% to compile a preprint version, e.g., for submission to arXiv, add add the
% [preprint] option:
%     \usepackage[preprint]{neurips_2021}

% to compile a camera-ready version, add the [final] option, e.g.:
%     \usepackage[final]{neurips_2021}

% to avoid loading the natbib package, add option nonatbib:
%    \usepackage[nonatbib]{neurips_2021}

\usepackage[utf8]{inputenc} % allow utf-8 input
\usepackage[T1]{fontenc}    % use 8-bit T1 fonts
\usepackage{hyperref}       % hyperlinks
\usepackage{url}            % simple URL typesetting
\usepackage{booktabs}       % professional-quality tables
\usepackage{amsfonts}       % blackboard math symbols
\usepackage{nicefrac}       % compact symbols for 1/2, etc.
\usepackage{microtype}      % microtypography
\usepackage{xcolor}         % colors
\usepackage{amsthm}
\usepackage{amsmath}
\newtheorem{mytheorem}{Theorem}
\newtheorem{definition}[mytheorem]{Definition}
\newtheorem{myremark}[mytheorem]{Remark}
\newtheorem{mylemma}[mytheorem]{Lemma}
\newtheorem{myquestion}[mytheorem]{Question}
\newtheorem{myexample}[mytheorem]{Example}
\newtheorem{myproposition}[mytheorem]{Proposition}
\newtheorem{mycorollary}[mytheorem]{Corollary}
\newtheorem{mydefinition}[mytheorem]{Definition}
\newtheorem{myprinciple}[mytheorem]{Principle}
\def\RC{\textrm{RC}}
\def\A{\mathcal{A}}
\def\O{\mathcal{O}}
\def\RND{\textrm{RND}}
\def\obs{\textrm{obs}}
\def\rwd{\textrm{rwd}}

\title{Appendix to:
Extending Environments To Measure Self-Reflection In Reinforcement Learning}

% The \author macro works with any number of authors. There are two commands
% used to separate the names and addresses of multiple authors: \And and \AND.
%
% Using \And between authors leaves it to LaTeX to determine where to break the
% lines. Using \AND forces a line break at that point. So, if LaTeX puts 3 of 4
% authors names on the first line, and the last on the second line, try using
% \AND instead of \And before the third author name.

\author{
  Samuel A.~Alexander\\
  The U.S.\ Securities and Exchange Commission\\
  New York, NY 10281\\
  \texttt{samuelallenalexander@gmail.com}\\
  \And
  Oscar Martinez\\
  The U.S.\ Securities and Exchange Commission\\
  New York, NY 10281\\
  \And
  Kevin Compher\\
  The U.S.\ Securities and Exchange Commission\\
  Washington, DC 20549\\
  \And
  Michael Castaneda\\
  First Derivatives\\
  New York, NY 10006\\
}

\begin{document}

\maketitle

\appendix

\section{Appendix}

As promised, we will now prove Proposition 9.

\begin{proof}[Proof of Proposition 9]
  To simplify the proof, we adopt the following notational convention: for any percept $x_1$,
  even if $y_1$ is not defined, we will write $x_1y_1\ldots x_1$ for $\langle x_1\rangle$.
  With this convention, the definition of $\pi_{\RC}$ simplifies to:
  \[
    \pi_{\RC}(x_1y_1\ldots x_n) =
    \begin{cases}
      \pi(x_1y_1\ldots x_n) & \mbox{if $y_i=\pi_{\RC}(x_1y_1\ldots x_i)$ for all $1\leq i<n$,}\\
      \pi(\langle x_1\rangle) & \mbox{otherwise}
    \end{cases}
  \]
  (since, when $n=1$, $\pi(x_1y_1\ldots x_n)$ and $\pi(\langle x_1\rangle)$ are the same thing).

  Let $D$ be the set of all sequences on which $\pi_{\RC}$ is defined, so, using the above
  convention, $D$ is
  the set of all sequences $x_1y_1\ldots x_n$ (each $x_i$ a percept, each $y_i$ an action).

  \item
  (Part 1) Define $\rho$ on $D$ by
  \[
    \rho(x_1y_1\ldots x_n) =
    \begin{cases}
      \pi(x_1y_1\ldots x_n) & \mbox{if $y_i=\pi(x_1y_1\ldots x_i)$ for all $1\leq i<n$,}\\
      \pi(\langle x_1\rangle) & \mbox{otherwise.}
    \end{cases}
  \]
  We wish to show that $\rho=\pi_{\RC}$, which will prove part 1 of
  Proposition 9.
  We will prove by induction that for each $x_1y_1\ldots x_n\in D$,
  $\rho(x_1y_1\ldots x_n)=\pi_{\RC}(x_1y_1\ldots x_n)$.
  The base case is trivial:
  $\rho(\langle x_1\rangle)=\pi(\langle x_1\rangle)=\pi_{\RC}(\langle x_1\rangle)$
  by definition of $\rho$ and $\pi_{\RC}$. For the induction step, assume $n>1$,
  and assume the claim holds for all shorter sequences
  in $D$.

  Case 1: Assume ($*$) for all $1\leq i<n$, $y_i=\pi(x_1y_1\ldots x_i)$.
  We claim that for all $1\leq i<n$, $y_i=\rho(x_1y_1\ldots x_i)$.
  To see this, choose any $1\leq i<n$. Then for all $1\leq j<i$,
  we must have $y_j=\pi(x_1y_1\ldots x_j)$ because otherwise $j$ would
  be a counterexample to ($*$). Thus
  \begin{align*}
    \rho(x_1y_1\ldots x_i) &= \pi(x_1y_1\ldots x_i) &\mbox{(By definition of $\rho$)}\\
    &= y_i, &\mbox{(By $*$)}
  \end{align*}
  proving the claim. Now, since we have proved that for all $1\leq i<n$,
  $y_i=\rho(x_1y_1\ldots x_i)$, and since our induction hypothesis is that for
  all such $i$,
  $\rho(x_1y_1\ldots x_i)=\pi_{\RC}(x_1y_1\ldots x_i)$,
  we may conclude that for all $1\leq i<n$, $y_i=\pi_{\RC}(x_1y_1\ldots x_i)$.
  Thus $\pi_{\RC}(x_1y_1\ldots x_n)=\pi(x_1y_1\ldots x_n)=\rho(x_1y_1\ldots x_n)$
  as desired.

  Case 2: Assume there is some $1\leq i<n$ such that
  $y_i\not=\pi(x_1y_1\ldots x_i)$.
  We may choose $i$ as small as possible.
  Thus, for all $1\leq j<i$, $y_j=\pi(x_1y_1\ldots x_j)$.
  By similar logic as in Case 1, it follows that for all
  $1\leq j<i$, $y_j=\rho(x_1y_1\ldots x_j)$.
  Our induction hypothesis says that for each such $j$,
  $\rho(x_1y_1\ldots x_j)=\pi_{\RC}(x_1y_1\ldots x_j)$.
  So for all $1\leq j<i$, $y_j=\pi_{\RC}(x_1y_1\ldots x_j)$.
  By definition of $\pi_{\RC}$, this means
  $\pi_{\RC}(x_1y_1\ldots x_i)=\pi(x_1y_1\ldots x_i)$.
  But $y_i\not=\pi(x_1y_1\ldots x_i)$, so therefore
  $y_i\not=\pi_{\RC}(x_1y_1\ldots x_i)$.
  Thus, since $1\leq i<n$, by definition of $\pi_{\RC}$,
  $\pi_{\RC}(x_1y_1\ldots x_n)=\pi(\langle x_1\rangle)$.
  Likewise, since $1\leq i<n$, by definition of $\rho$,
  $\rho(x_1y_1\ldots x_n)=\pi(\langle x_1\rangle)$.
  So $\rho(x_1y_1\ldots x_n)=\pi_{\RC}(x_1y_1\ldots x_n)$ as desired.

  \item
  (Part 2)
  We must show that $\pi_{\RC}$ is deterministic.
  Let $x_1y_1\ldots x_n\in D$, we must show that any time we compute
  $\pi_{\RC}(x_1y_1\ldots x_n)$, we get the same result.
  We prove this by induction on $n$.
  For the base case, if $n=1$, $\pi_{\RC}(x_1y_1\ldots x_n)=\pi(\langle x_1\rangle)$
  yields the same result every time because $\pi$ is deterministic.
  For the induction step, assume $n>1$ and that the claim holds for all smaller
  sequences.
  When we compute $\pi_{\RC}(x_1y_1\ldots x_n)$,
  first we compute $\pi_{\RC}(x_1y_1\ldots x_i)$ for $i=1,\ldots,n-1$,
  and check whether the results are $y_1,\ldots,y_{n-1}$, respectively.
  Each of these computations always has the same outcome, by the induction hypothesis.
  So, every time we check whether or not each $y_i=\pi_{\RC}(x_1y_1\ldots x_i)$ for
  all $1\leq i<n$, we get the same answer.
  If that answer is ``yes'', then we finally output $\pi(x_1y_1\ldots x_n)$ (which is
  deterministic since $\pi$ is deterministic). Otherwise, we finally output
  $\pi(\langle x_1\rangle)$ (which is deterministic since $\pi$ is deterministic).

  \item
  (Part 3)
  We will show by induction on $n$ that for all $x_1y_1\ldots x_n\in D$,
  $\pi_{\RC}(x_1y_1\ldots x_n)=(\pi_{\RC})_{\RC}(x_1y_1\ldots x_n)$. For the base case,
  this is trivial, both evaluate to $\pi(\langle x_1\rangle)$.
  For the induction step, assume $n>1$ and that the claim holds for all shorter sequences.

  Case 1: $y_i=\pi_{\RC}(x_1y_1\ldots x_i)$ for all $1\leq i<n$.
  Then by induction, $y_i=(\pi_{\RC})_{\RC}(x_1y_1\ldots x_i)$ for all $1\leq i<n$.
  By definition of $(\pi_{\RC})_{\RC}$, this means
  $(\pi_{\RC})_{\RC}(x_1y_1\ldots x_n)=\pi_{\RC}(x_1y_1\ldots x_n)$, as desired.

  Case 2: There is some $1\leq i<n$ such that $y_i\not=\pi_{\RC}(x_1y_1\ldots x_i)$.
  By induction, $y_i\not=(\pi_{\RC})_{\RC}(x_1y_1\ldots x_i)$.
  Thus, $(\pi_{\RC})_{\RC}(x_1y_1\ldots x_n)=\pi_{\RC}(\langle x_1\rangle)
  =\pi(\langle x_1\rangle)$, which equals $\pi_{\RC}(x_1y_1\ldots x_n)$
  since $y_i\not=\pi_{\RC}(x_1y_1\ldots x_i)$ and $i<n$.

  \item
  (Part 4)
  Let $\mu$ be a deterministic non-extended environment, let $x_1y_1x_2y_2\ldots$ be the
  result of $\pi$ interacting with $\mu$, and let $x'_1y'_1x'_2y'_2\ldots$ be the
  result of $\pi_{\RC}$ interacting with $\mu$. We will show by induction that each
  $x_n=x'_n$ and each $y_n=y'_n$.
  For the base case, $x_1=x'_1=\mu(\langle\rangle)$ (the environment's initial percept
  does not depend on the agent), and therefore
  $y_1=\pi(\langle x_1\rangle)=\pi(\langle x'_1\rangle)=y'_1$. For the induction step,
  \begin{align*}
    x_{n+1} &= \mu(x_1y_1\ldots x_ny_n)\\
      &= \mu(x'_1y'_1\ldots x'_ny'_n) &\mbox{(By induction)}\\
      &= x'_{n+1},\\
    y_{n+1} &= \pi(x_1y_1\ldots x_{n+1})\\
      &= \pi(x'_1y'_1\ldots x'_{n+1}), &\mbox{(Induction plus $x_{n+1}=x'_{n+1}$)}\\
  \end{align*}
  and the latter is $\pi_{\RC}(x'_1y'_1\ldots x'_{n+1})$
  because for all $1\leq i<n$, $y'_i=\pi_{\RC}(x'_1y'_1\ldots x'_i)$
  since $x'_1y'_1x'_2y'_2\ldots$ is the result of $\pi_{\RC}$ interacting with $\mu$.
  And finally, $\pi_{\RC}(x'_1y'_1\ldots x'_{n+1})$ is $y'_{n+1}$,
  so $y_{n+1}=y'_{n+1}$.
\end{proof}
\end{document}