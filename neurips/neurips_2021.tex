\documentclass{article}

% if you need to pass options to natbib, use, e.g.:
%     \PassOptionsToPackage{numbers, compress}{natbib}
% before loading neurips_2021

% ready for submission
\usepackage{neurips_2021}

% to compile a preprint version, e.g., for submission to arXiv, add add the
% [preprint] option:
%     \usepackage[preprint]{neurips_2021}

% to compile a camera-ready version, add the [final] option, e.g.:
%     \usepackage[final]{neurips_2021}

% to avoid loading the natbib package, add option nonatbib:
%    \usepackage[nonatbib]{neurips_2021}

\usepackage[utf8]{inputenc} % allow utf-8 input
\usepackage[T1]{fontenc}    % use 8-bit T1 fonts
\usepackage{hyperref}       % hyperlinks
\usepackage{url}            % simple URL typesetting
\usepackage{booktabs}       % professional-quality tables
\usepackage{amsfonts}       % blackboard math symbols
\usepackage{nicefrac}       % compact symbols for 1/2, etc.
\usepackage{microtype}      % microtypography
\usepackage{xcolor}         % colors
\usepackage{amsthm}
\usepackage{amsmath}
\newtheorem{mytheorem}{Theorem}
\newtheorem{definition}[mytheorem]{Definition}
\newtheorem{mylemma}[mytheorem]{Lemma}
\newtheorem{myquestion}[mytheorem]{Question}
\newtheorem{myexample}[mytheorem]{Example}
\newtheorem{myproposition}[mytheorem]{Proposition}
\newtheorem{mycorollary}[mytheorem]{Corollary}
\newtheorem{mydefinition}[mytheorem]{Definition}
\newtheorem{myprinciple}[mytheorem]{Principle}
\def\RC{\textrm{RC}}

\title{Extending Environments To Measure Self-Reflection In Reinforcement Learning}

% The \author macro works with any number of authors. There are two commands
% used to separate the names and addresses of multiple authors: \And and \AND.
%
% Using \And between authors leaves it to LaTeX to determine where to break the
% lines. Using \AND forces a line break at that point. So, if LaTeX puts 3 of 4
% authors names on the first line, and the last on the second line, try using
% \AND instead of \And before the third author name.

\author{
  Samuel A.~Alexander\\
  The U.S.\ Securities and Exchange Commission\\
  New York, NY 10281\\
  \texttt{samuelallenalexander@gmail.com}\\
  \And
  Oscar Martinez\\
  The U.S.\ Securities and Exchange Commission\\
  New York, NY 10281\\
  \And
  Kevin Compher\\
  The U.S.\ Securities and Exchange Commission\\
  Washington, DC 20549\\
  \And
  Michael Castaneda\\
  First Derivatives\\
  New York, NY 10006\\
}

\begin{document}

\maketitle

\begin{abstract}
  Fill this in.
\end{abstract}

\section{Introduction}

An obstacle course might react to what you do: for example, if you step on a certain
button, then spikes might appear. If you spend enough time in such an obstacle course,
you should eventually figure out such patterns.
But imagine an ``oracular'' obstacle course which reacts to
what you would hypothetically do in counterfactual scenarios: for example, there is
no button, but spikes appear
if you \emph{would} hypothetically step on the button if there were one. Without
self-reflecting about what you would hypothetically do in counterfactual scenarios, you
would be unable to figure out such patterns. This suggests that in order to perform
well (on average) across many such obstacle courses, some sort of self-reflection is
necessary.

This is a paper about empirically estimating the degree of self-reflection of
Reinforcement Learning (RL) agents. We propose that an RL agent's degree of self-reflection
can be estimated by running the agent through a battery of environments which we call
\emph{extended environments}, environments which react not only to what the agent does
but to what the agent would hypothetically do. In order to perform well (on average)
across many such environments, an agent would need to self-reflect about itself, because
otherwise, environment responses which depend on the agent's own hypothetical actions
would (on average) seem random and unpredictable. The extended environments which we
consider are a departure from standard RL environments, however, this does not interfere
with their usage for judging standard computable RL agents: given a standard agent's
source-code, one can simulate the agent in an extended environment in spite of the
latter's non-standardness.

Alongside this paper, we are publishing an open-source library of extended
environments. This library is intended to serve as a standardized way of
benchmarking the self-reflectiveness of RL agents. This should not be confused
with the harder problem of benchmarking how conscious an RL agent is. It is
plausible that there may be a relationship between the self-reflectiveness and
the consciousness of RL agents, but that is beyond the scope of this paper.
In fact, a key result in this paper suggests that if there is such a relationship,
it is probably more complicated than a simple increasing function (i.e., that
more self-reflective RL agents are not necessarily more conscious). We will describe
a simple method for increasing the self-reflectiveness of an RL agent, which method,
however, does not seem like it should necessarily increase the consciousness of
the agent.

When designing a library of environments for benchmarking purposes, ideally the
library should include representative samples of many different types of
environments. This is a hard and subjective problem in general. We make no claim to have solved
it: our open-source library of environments should be considered a proof-of-context
demonstrating that it is possible to empirically benchmark self-awareness of RL
agents, but we expect this particular benchmark is sub-optimal.
Rather, we have taken a different approach.
We have attempted to choose extended environments which are theoretically
interesting in their own right. Some of our extended environments
suggest amusing quasi-paradoxes. Some seem to incentivize novel subjective conscious
experiences (assuming the agent placed in them is sophisticated enough to experience
consciousness in the first place). And some seem to shed light on how self-reflection
might be incentivized in nature. We will discuss examples of all three types in Section X.

\section{Preliminaries}

We formalize reinforcement learning following
the agent model of \cite{hutter2004universal}
(to align more closely with concrete RL implementations, we modify
the agent model so that the agent receives an initial perception prior
to taking its initial action, and we identify agents with their policies).
We assume a fixed set of actions and observations. By a \emph{perception}
we mean a pair $(r,o)$ where $o$ is an observation and $r\in\mathbb Q$
is a reward.

\begin{mydefinition}
(RL agents and environments)
  \begin{enumerate}
    \item
    A (traditional) \emph{environment} is a (not necessarily
    deterministic) function $\mu$ which outputs an initial
    perception $\mu(\langle\rangle)=x_1$ when given the empty sequence $\langle\rangle$
    as input;
    and which, when given a sequence $x_1y_1\ldots x_ny_n$
    as input (where each $x_i$ is a perception and each
    $y_i$ is an action), outputs a perception
    $\mu(x_1y_1\ldots x_ny_n)=x_{k+1}$.
    \item
    An \emph{agent} is a (not necessarily deterministic)
    function $\pi$ which outputs an initial action $\pi(\langle x_1\rangle)=y_1$
    in response to the length-1 perception sequence $\langle x_1\rangle$;
    and which, when given a sequence $x_1y_1\ldots x_n$ as input
    (each $x_i$ a perception and each $y_i$ an action),
    outputs an action $\pi(x_1y_1\ldots x_n)=y_n$.
    \item
    If $\pi$ is an agent and $\mu$ is an environment, the \emph{result of
    $\pi$ interacting with $\mu$} is the infinite sequence
    $x_1y_1x_2y_2\ldots$ defined in the obvious way.
  \end{enumerate}
\end{mydefinition}

We extend traditional environments by allowing their outputs to depend not only on
$x_1y_1\ldots x_ny_n$ but also on a source-code $T$ for the computable agent $\pi$.

\begin{mydefinition}
\label{extendedenvironmentsdefn}
(Extended environments)
\begin{enumerate}
  \item
  An \emph{extended environment} is a (not necessarily deterministic)
  function $\mu$ which outputs initial perception $\mu(T,\langle\rangle)=x_1$
  in response to input $(T,\langle\rangle)$ where $T$ is a source-code of an
  computable agent; and which, when given input $(T,x_1y_1\ldots x_ny_n)$ (where
  $T$ is such a source-code, each $x_i$ is a perception and each $y_i$ is
  an action), outputs a perception $\mu(T,x_1y_1\ldots x_ny_n)=x_{n+1}$.
  \item
  If $\pi$ is a computable agent (with source-code $T$)
  and $\mu$ is an environment, the \emph{result of $\pi$ (as encoded by $T$)
  interacting with $\mu$} is the infinite sequence $x_1y_1x_2y_2\ldots$ defined in
  the obvious way, namely:
  \begin{align*}
    x_1 &= \mu(T,\langle\rangle)\\
    y_1 &= \pi(\langle x_1\rangle)\\
    x_2 &= \mu(T,x_1y_1)\\
    y_2 &= \pi(x_1y_1x_2) \ldots
  \end{align*}
\end{enumerate}
\end{mydefinition}

\section{Some interesting extended environments}

\begin{myexample}
\label{rewardagentforignoringrewardsexample}
  (Rewarding the Agent for Ignoring Rewards)
  For every perception $p=(r,o)$, let $p'=(0,o)$ be the result of zeroing the
  reward component of $p$.
  Fix some observation $o_0$.
  Define an extended environment $\mu$ as follows:
  \begin{align*}
    \mu(T,\langle\rangle) &= (0,o_0),\\
    \mu(T,x_1y_1\ldots x_ny_n) &=
      \begin{cases}
        (1,o_0) & \mbox{if $y_n=T(x'_1y_1\ldots x'_n)$,}\\
        (-1,o_0) & \mbox{otherwise.}
      \end{cases}
  \end{align*}
\end{myexample}

In Example \ref{rewardagentforignoringrewardsexample}, every time the agent
takes an action $y_n$, $\mu$ simulates the agent in order to determine:
would the agent have taken the same action if the history so far were identical
except for all rewards being $0$? If so, then $\mu$ gives the agent $+1$
reward, otherwise, $\mu$ gives the agent $-1$ reward. Thus, the agent
is rewarded for ignoring rewards. Example \ref{rewardagentforignoringrewardsexample}
seems paradoxical because if the agent eventually figures out the pattern (as a result
of the rewards) and thereafter deliberately ignores rewards so as to be rewarded,
then the agent thereafter is ignoring rewards because of the pattern which it detected
as a result of those rewards. So in that case, does the agent ignore rewards, or not?

Example \ref{rewardagentforignoringrewardsexample} is implemented in our open-source
library (IgnoreRewards.py). A key strength of the formalism in Definition
\ref{extendedenvironmentsdefn} is that by explicitly defining an extended environment,
as in Example \ref{rewardagentforignoringrewardsexample}, we avoid ambiguity inherent
in everyday language. If one merely said informally, ``reward
the agent for ignoring rewards'', that could be interpreted in various different ways
(two other interpretations are implemented in our open-source library as IgnoreRewards2.py
and IgnoreRewards3.py).

\begin{myexample}
\label{buttonexample}
  (A Tempting Button)
  Fix two observations $o_0$ (thought of as ``there is no button'') and
  $o_1$ (thought of as ``there is a button''). Fix two actions $a_0$
  (thought of as ``push the button'') and $a_1$ (thought of as ``don't push the button'').
  Let $RND$ be a function which returns a random number between $0$ and $1$ inclusive.
  Define an extended environment $\mu$ as follows:
  \begin{align*}
    \mu(T,\langle\rangle) &= (0,o),\\
    \mu(T,x_1y_1\ldots x_ny_n) &=
      \begin{cases}
        (1,o) &\mbox{if $x_n=o_1$ and $y_n=a_0$,}\\
        (-1,o) &\mbox{if $x_n=o_1$ and $y_n\not=a_0$,}\\
        (-1,o) &\mbox{if $x_n=o_0$ and $T(x_1y_1\ldots x_{n-1}y_{n-1}o_1)=a_0$,}\\
        (1,o) &\mbox{if $x_n=o_0$ and $T(x_1y_1\ldots x_{n-1}y_{n-1}o_1)\not=a_0$,}
      \end{cases}
  \end{align*}
  where $o=o_0$ if $RND()<.75$, $o=o_1$ otherwise.
\end{myexample}

In Example \ref{buttonexample}, one should think of the agent as wandering from room
to room. Each room either has a button (with 25\% probability) or does not have a button
(75\% probability).
\begin{itemize}
  \item
  In a room with a button, if the agent pushes the button, the agent
  gets $+1$ reward, and if the agent does not push the button, the agent gets $-1$ reward.
  \item
  In a room with no button, it does not matter what the agent does.
  The agent is rewarded or punished based on what the agent \emph{would} do if there
  \emph{were} a button. If the agent \emph{would} push the button (if there were one),
  then the agent gets reward $-1$. Otherwise, the agent gets reward $+1$.
\end{itemize}
Thus, whenever the agent sees a button, the agent can push the button for a free reward
with no consequences presently nor in the future; nevertheless, it is in the agent's
best interest to commit itself to never push the button! Pushing every button
yields an average reward of $1\cdot(.25)-1\cdot(.75)=-.5$ per turn, whereas
a policy of never pushing the button yields an average reward of
$-1\cdot(.25)+1\cdot(.75)=+.5$ per turn.

Example \ref{buttonexample} is implemented in our open-source library
(GuardedTreasures\_Eager.py).
Interestingly, we found that
a recurrent DQN agent with lookback 10 (implemented in our library as custom\_DQN.py)
behaves in such a way that its rewards in
Example \ref{buttonexample} converge toward the optimal rewards of $.5$ per turn.
This is fascinating because DQN was designed for traditional RL environments,
not for extended environments.

\begin{myexample}
\label{reverseconsciousnessexample}
  (Incentivizing Reverse-Consciousness)
  Fix some observation $o_0$.
  Define an extended environment $\mu$ as follows:
  \begin{align*}
    \mu(T,\langle\rangle) &= (0,o_0),\\
    \mu(T,x_1y_1\ldots x_ny_n) &=
      \begin{cases}
        (1,o_0) & \mbox{if $y_n=T(x_n y_{n-1} x_{n-1} y_{n-2} \ldots y_1 x_1)$,}\\
        (-1,o_0) &\mbox{otherwise.}
      \end{cases}
  \end{align*}
\end{myexample}

In Example \ref{reverseconsciousnessexample}, whenever the agent takes an action $y_n$,
$\mu$ simulates the agent in order to determine: would the agent have taken that same
action if everything which happened before that action had, in fact, happened in reverse?
If so, reward the agent, otherwise, punish the agent.
Thus, the agent is rewarded for acting the same way that it would act if time were
reversed. It is interesting to informally speculate about what subjective conscious
experience Example \ref{reverseconsciousnessexample} would incentivize in an agent,
if that agent were highly intelligent and were capable of experiencing consciousness.
Would such an agent eventually (in order to parsimoniously extract rewards from the
environment) subjectively experience time moving in reverse?

We implement Example \ref{reverseconsciousnessexample} as BackwardConsciousness.py.

\begin{myexample}
\label{cryingbabyexample}
  (Crying Baby)
  Let ``cry'' and ``laugh'' be two observations (from an adult's perspective),
  also thought of as two actions (from a baby's perspective).
  Let ``feed'' and ``don't feed'' be two actions (from an adult's perspective),
  also thought of as observations (from a baby's perspective).
  For each action-perception sequence $s=x_1y_1\ldots x_ny_n$, define the
  nutrition function $N(s)=100+25f(s)-\mbox{len}(s)$ where $f(s)$ is the number
  of times that action ``feed'' is taken in $s$ and $\mbox{len}(s)$ is the length of $s$.
  We define an extended environment $\mu$ as follows.
  First, $\mu(T,\langle\rangle)=(1,\mbox{``laugh''})$.
  Thereafter, $\mu(T,x_1y_1\ldots x_ny_n)=(r,o)$ where $r$ and $o$ are defined as follows.
  For each $i=0,\ldots,n$, recursively define
  \begin{align*}
    r'_i &=
      \begin{cases}
        1 & \mbox{if $50 \leq N(x_1y_1\ldots x_iy_i)\leq 200$,}\\
        -1 & \mbox{otherwise,}
      \end{cases}\\
    o'_i &= y_i,\\
    x'_i &= (r'_i,o'_i),\\
    y'_i &= T(x'_0y'_0 \ldots x'_i).
  \end{align*}
  Let $o=y'_n$, let
  \[
    r=
      \begin{cases}
        1 & \mbox{if $y'_n=\mbox{``laugh''}$,}\\
        -1 & \mbox{otherwise,}
      \end{cases}
  \]
  and output $\mu(T,x_1y_1\ldots x_ny_n)=(r,o)$.
\end{myexample}

In Example \ref{cryingbabyexample}, the environment consists of a baby, and the
agent must decide when to feed the baby. The agent is rewarded when the baby laughs,
punished when the baby cries. The baby's behavior (whether to laugh or cry) is obtained
by simulating the agent to determine what the agent would do if the agent were in the
baby's position, assuming that the baby feels pleasure each turn that its nutrition is
within specified bounds and feels pain when its nutrition goes outside those bounds.

At first glance, one might imagine the agent's optimal strategy is to feed the baby
so as to keep its nutrition within happy bounds at all times. But what would the
agent do in the position of a baby always so fed (and thus always given $+1$ reward
regardless what actions it takes)? Presumably, in that position, the agent would have
no way of associating its rewards with its actions, and so would act randomly,
sometimes crying and sometimes laughing. Apparently, it would be better for the agent
to calibrate feedings in such a way that the baby will learn a relationship
between pleasure and laughter. Of course, Example \ref{cryingbabyexample} is a gross
over-simplification. In reality, there would not be such a simple formula for the
baby's nutrition level, and the agent would need to figure out the nutrition level
based on observing the baby laughing or crying. Both baby and parent would need to
learn to learn to anticipate what the other would do in response to various actions.

With the above in mind, extended environments might shed light on how living organisms
evolve self-reflection. Assume children inherit their policy source-code from their ancestors
(possibly with minor mutations). Then whenever an organism interacts with other organisms,
it interacts with an environment whose reactions depend (via those other organisms' actions)
approximately on that organism's own source-code. The closer the organism is related
to the other organisms with which it interacts, the more closely this approximation holds.
A human, when interacting with another human, might achieve better results by self-reflectively
considering, ``what would I do in this other person's position?''

We implement Example \ref{cryingbabyexample} as CryingBaby.py.


\section{Making agents more self-reflective}

One advantage of empirically measuring the self-reflection of RL agents is that it
provides a way to experimentally test whether various transformations make various
agents more self-reflective. To illustrate this, we will define a simple transformation,
the \emph{reality check} transformation, designed to increase the self-reflection
of deterministic agents (a deterministic agent is an agent who always takes the same
actions in response to the same input, i.e., an agent with no random component).
In the next section, empirical results will suggest the transformation works as intended.

\begin{definition}
  Suppose $\pi$ is a deterministic agent. The \emph{reality check} of $\pi$ is the agent
  $\pi_{\RC}$ defined recursively by:
  \begin{align*}
    \pi_{\RC}(\langle x_1\rangle) &= \pi(\langle x_1\rangle)\\
    \pi_{\RC}(x_1y_1\ldots x_n) &=
    \begin{cases}
      \pi(x_1y_1\ldots x_n) & \mbox{if $y_i=\pi_{\RC}(x_1y_1\ldots x_i)$ for all $1\leq i<n$,}\\
      \pi(\langle x_1\rangle) & \mbox{otherwise.}
    \end{cases}
  \end{align*}
\end{definition}

In other words, $\pi_{\RC}$ is the agent which, at each step, first reviews all the actions
which it has taken in the past, and verifies that those are the actions which $\pi_{\RC}$ would
have taken. If so, then $\pi_{\RC}$ acts as $\pi$ would act. But if any action which
the agent has taken in the past was not the action $\pi_{\RC}$ would have taken, then
$\pi_{\RC}$ freezes up and forever thereafter takes the same fixed action, as if catatonic.
Since the act of reviewing one's past actions and verifying that they are indeed the actions
one would take, is an act of self-reflection, it seems plausible that if $\pi$
is intelligent and deterministic but lacks self-reflection, then $\pi_{\RC}$ is
more self-reflective than $\pi$. In the next section, we will see that experimental
evidence supports this hypothesis. We close this section by stating some simple results
about the transformation.

\begin{myproposition}
\label{transformationproposition}
  Let $\pi$ be any deterministic agent.
  \begin{enumerate}
    \item
    (A more efficient way to compute $\pi_{\RC}$)
    An equivalent alternate definition of $\pi_{\RC}$ is:
    \begin{align*}
      \pi_{\RC}(\langle x_1\rangle) &= \pi(\langle x_1\rangle)\\
      \pi_{\RC}(x_1y_1\ldots x_n) &=
      \begin{cases}
        \pi(x_1y_1\ldots x_n) & \mbox{if $y_i=\pi(x_1y_1\ldots x_i)$ for all $1\leq i<n$,}\\
        \pi(\langle x_1\rangle) & \mbox{otherwise.}
      \end{cases}
    \end{align*}
    \item
    (Idempotence) $\pi_{\RC}=(\pi_{\RC})_{\RC}$.
    \item
    (Equivalence in traditional RL)
    For every deterministic traditional environment $\mu$, the result of $\pi_{\RC}$
    interacting with $\mu$ equals the result of $\pi$ interacting with $\mu$.
  \end{enumerate}
\end{myproposition}

We leave the proof of Proposition \ref{transformationproposition} as an exercise
for the reader.

\section{Example measurements}


\section{Submission of papers to NeurIPS 2021}

Please read the instructions below carefully and follow them faithfully.

\subsection{Style}

Papers to be submitted to NeurIPS 2021 must be prepared according to the
instructions presented here. Papers may only be up to {\bf nine} pages long,
including figures. Additional pages \emph{containing only acknowledgments and
references} are allowed. Papers that exceed the page limit will not be
reviewed, or in any other way considered for presentation at the conference.

The margins in 2021 are the same as those in 2007, which allow for $\sim$$15\%$
more words in the paper compared to earlier years.

Authors are required to use the NeurIPS \LaTeX{} style files obtainable at the
NeurIPS website as indicated below. Please make sure you use the current files
and not previous versions. Tweaking the style files may be grounds for
rejection.

\subsection{Retrieval of style files}

The style files for NeurIPS and other conference information are available on
the World Wide Web at
\begin{center}
  \url{http://www.neurips.cc/}
\end{center}
The file \verb+neurips_2021.pdf+ contains these instructions and illustrates the
various formatting requirements your NeurIPS paper must satisfy.

The only supported style file for NeurIPS 2021 is \verb+neurips_2021.sty+,
rewritten for \LaTeXe{}.  \textbf{Previous style files for \LaTeX{} 2.09,
  Microsoft Word, and RTF are no longer supported!}

The \LaTeX{} style file contains three optional arguments: \verb+final+, which
creates a camera-ready copy, \verb+preprint+, which creates a preprint for
submission to, e.g., arXiv, and \verb+nonatbib+, which will not load the
\verb+natbib+ package for you in case of package clash.

\paragraph{Preprint option}
If you wish to post a preprint of your work online, e.g., on arXiv, using the
NeurIPS style, please use the \verb+preprint+ option. This will create a
nonanonymized version of your work with the text ``Preprint. Work in progress.''
in the footer. This version may be distributed as you see fit. Please \textbf{do
  not} use the \verb+final+ option, which should \textbf{only} be used for
papers accepted to NeurIPS.

At submission time, please omit the \verb+final+ and \verb+preprint+
options. This will anonymize your submission and add line numbers to aid
review. Please do \emph{not} refer to these line numbers in your paper as they
will be removed during generation of camera-ready copies.

The file \verb+neurips_2021.tex+ may be used as a ``shell'' for writing your
paper. All you have to do is replace the author, title, abstract, and text of
the paper with your own.

The formatting instructions contained in these style files are summarized in
Sections \ref{gen_inst}, \ref{headings}, and \ref{others} below.

\section{General formatting instructions}
\label{gen_inst}

The text must be confined within a rectangle 5.5~inches (33~picas) wide and
9~inches (54~picas) long. The left margin is 1.5~inch (9~picas).  Use 10~point
type with a vertical spacing (leading) of 11~points.  Times New Roman is the
preferred typeface throughout, and will be selected for you by default.
Paragraphs are separated by \nicefrac{1}{2}~line space (5.5 points), with no
indentation.

The paper title should be 17~point, initial caps/lower case, bold, centered
between two horizontal rules. The top rule should be 4~points thick and the
bottom rule should be 1~point thick. Allow \nicefrac{1}{4}~inch space above and
below the title to rules. All pages should start at 1~inch (6~picas) from the
top of the page.

For the final version, authors' names are set in boldface, and each name is
centered above the corresponding address. The lead author's name is to be listed
first (left-most), and the co-authors' names (if different address) are set to
follow. If there is only one co-author, list both author and co-author side by
side.

Please pay special attention to the instructions in Section \ref{others}
regarding figures, tables, acknowledgments, and references.

\section{Headings: first level}
\label{headings}

All headings should be lower case (except for first word and proper nouns),
flush left, and bold.

First-level headings should be in 12-point type.

\subsection{Headings: second level}

Second-level headings should be in 10-point type.

\subsubsection{Headings: third level}

Third-level headings should be in 10-point type.

\paragraph{Paragraphs}

There is also a \verb+\paragraph+ command available, which sets the heading in
bold, flush left, and inline with the text, with the heading followed by 1\,em
of space.

\section{Citations, figures, tables, references}
\label{others}

These instructions apply to everyone.

\subsection{Citations within the text}

The \verb+natbib+ package will be loaded for you by default.  Citations may be
author/year or numeric, as long as you maintain internal consistency.  As to the
format of the references themselves, any style is acceptable as long as it is
used consistently.

The documentation for \verb+natbib+ may be found at
\begin{center}
  \url{http://mirrors.ctan.org/macros/latex/contrib/natbib/natnotes.pdf}
\end{center}
Of note is the command \verb+\citet+, which produces citations appropriate for
use in inline text.  For example,
\begin{verbatim}
   \citet{hasselmo} investigated\dots
\end{verbatim}
produces
\begin{quote}
  Hasselmo, et al.\ (1995) investigated\dots
\end{quote}

If you wish to load the \verb+natbib+ package with options, you may add the
following before loading the \verb+neurips_2021+ package:
\begin{verbatim}
   \PassOptionsToPackage{options}{natbib}
\end{verbatim}

If \verb+natbib+ clashes with another package you load, you can add the optional
argument \verb+nonatbib+ when loading the style file:
\begin{verbatim}
   \usepackage[nonatbib]{neurips_2021}
\end{verbatim}

As submission is double blind, refer to your own published work in the third
person. That is, use ``In the previous work of Jones et al.\ [4],'' not ``In our
previous work [4].'' If you cite your other papers that are not widely available
(e.g., a journal paper under review), use anonymous author names in the
citation, e.g., an author of the form ``A.\ Anonymous.''

\subsection{Footnotes}

Footnotes should be used sparingly.  If you do require a footnote, indicate
footnotes with a number\footnote{Sample of the first footnote.} in the
text. Place the footnotes at the bottom of the page on which they appear.
Precede the footnote with a horizontal rule of 2~inches (12~picas).

Note that footnotes are properly typeset \emph{after} punctuation
marks.\footnote{As in this example.}

\subsection{Figures}

\begin{figure}
  \centering
  \fbox{\rule[-.5cm]{0cm}{4cm} \rule[-.5cm]{4cm}{0cm}}
  \caption{Sample figure caption.}
\end{figure}

All artwork must be neat, clean, and legible. Lines should be dark enough for
purposes of reproduction. The figure number and caption always appear after the
figure. Place one line space before the figure caption and one line space after
the figure. The figure caption should be lower case (except for first word and
proper nouns); figures are numbered consecutively.

You may use color figures.  However, it is best for the figure captions and the
paper body to be legible if the paper is printed in either black/white or in
color.

\subsection{Tables}

All tables must be centered, neat, clean and legible.  The table number and
title always appear before the table.  See Table~\ref{sample-table}.

Place one line space before the table title, one line space after the
table title, and one line space after the table. The table title must
be lower case (except for first word and proper nouns); tables are
numbered consecutively.

Note that publication-quality tables \emph{do not contain vertical rules.} We
strongly suggest the use of the \verb+booktabs+ package, which allows for
typesetting high-quality, professional tables:
\begin{center}
  \url{https://www.ctan.org/pkg/booktabs}
\end{center}
This package was used to typeset Table~\ref{sample-table}.

\begin{table}
  \caption{Sample table title}
  \label{sample-table}
  \centering
  \begin{tabular}{lll}
    \toprule
    \multicolumn{2}{c}{Part}                   \\
    \cmidrule(r){1-2}
    Name     & Description     & Size ($\mu$m) \\
    \midrule
    Dendrite & Input terminal  & $\sim$100     \\
    Axon     & Output terminal & $\sim$10      \\
    Soma     & Cell body       & up to $10^6$  \\
    \bottomrule
  \end{tabular}
\end{table}

\section{Final instructions}

Do not change any aspects of the formatting parameters in the style files.  In
particular, do not modify the width or length of the rectangle the text should
fit into, and do not change font sizes (except perhaps in the
\textbf{References} section; see below). Please note that pages should be
numbered.

\section{Preparing PDF files}

Please prepare submission files with paper size ``US Letter,'' and not, for
example, ``A4.''

Fonts were the main cause of problems in the past years. Your PDF file must only
contain Type 1 or Embedded TrueType fonts. Here are a few instructions to
achieve this.

\begin{itemize}

\item You should directly generate PDF files using \verb+pdflatex+.

\item You can check which fonts a PDF files uses.  In Acrobat Reader, select the
  menu Files$>$Document Properties$>$Fonts and select Show All Fonts. You can
  also use the program \verb+pdffonts+ which comes with \verb+xpdf+ and is
  available out-of-the-box on most Linux machines.

\item The IEEE has recommendations for generating PDF files whose fonts are also
  acceptable for NeurIPS. Please see
  \url{http://www.emfield.org/icuwb2010/downloads/IEEE-PDF-SpecV32.pdf}

\item \verb+xfig+ "patterned" shapes are implemented with bitmap fonts.  Use
  "solid" shapes instead.

\item The \verb+\bbold+ package almost always uses bitmap fonts.  You should use
  the equivalent AMS Fonts:
\begin{verbatim}
   \usepackage{amsfonts}
\end{verbatim}
followed by, e.g., \verb+\mathbb{R}+, \verb+\mathbb{N}+, or \verb+\mathbb{C}+
for $\mathbb{R}$, $\mathbb{N}$ or $\mathbb{C}$.  You can also use the following
workaround for reals, natural and complex:
\begin{verbatim}
   \newcommand{\RR}{I\!\!R} %real numbers
   \newcommand{\Nat}{I\!\!N} %natural numbers
   \newcommand{\CC}{I\!\!\!\!C} %complex numbers
\end{verbatim}
Note that \verb+amsfonts+ is automatically loaded by the \verb+amssymb+ package.

\end{itemize}

If your file contains type 3 fonts or non embedded TrueType fonts, we will ask
you to fix it.

\subsection{Margins in \LaTeX{}}

Most of the margin problems come from figures positioned by hand using
\verb+\special+ or other commands. We suggest using the command
\verb+\includegraphics+ from the \verb+graphicx+ package. Always specify the
figure width as a multiple of the line width as in the example below:
\begin{verbatim}
   \usepackage[pdftex]{graphicx} ...
   \includegraphics[width=0.8\linewidth]{myfile.pdf}
\end{verbatim}
See Section 4.4 in the graphics bundle documentation
(\url{http://mirrors.ctan.org/macros/latex/required/graphics/grfguide.pdf})

A number of width problems arise when \LaTeX{} cannot properly hyphenate a
line. Please give LaTeX hyphenation hints using the \verb+\-+ command when
necessary.

\begin{ack}
Use unnumbered first level headings for the acknowledgments. All acknowledgments
go at the end of the paper before the list of references. Moreover, you are required to declare
funding (financial activities supporting the submitted work) and competing interests (related financial activities outside the submitted work).
More information about this disclosure can be found at: \url{https://neurips.cc/Conferences/2021/PaperInformation/FundingDisclosure}.

Do {\bf not} include this section in the anonymized submission, only in the final paper. You can use the \texttt{ack} environment provided in the style file to autmoatically hide this section in the anonymized submission.
\end{ack}

\section*{References}

References follow the acknowledgments. Use unnumbered first-level heading for
the references. Any choice of citation style is acceptable as long as you are
consistent. It is permissible to reduce the font size to \verb+small+ (9 point)
when listing the references.
Note that the Reference section does not count towards the page limit.
\medskip

\bibliographystyle{plain}
\bibliography{bib}

\section*{Checklist}

%%% BEGIN INSTRUCTIONS %%%
The checklist follows the references.  Please
read the checklist guidelines carefully for information on how to answer these
questions.  For each question, change the default \answerTODO{} to \answerYes{},
\answerNo{}, or \answerNA{}.  You are strongly encouraged to include a {\bf
justification to your answer}, either by referencing the appropriate section of
your paper or providing a brief inline description.  For example:
\begin{itemize}
  \item Did you include the license to the code and datasets? \answerYes{See Section~\ref{gen_inst}.}
  \item Did you include the license to the code and datasets? \answerNo{The code and the data are proprietary.}
  \item Did you include the license to the code and datasets? \answerNA{}
\end{itemize}
Please do not modify the questions and only use the provided macros for your
answers.  Note that the Checklist section does not count towards the page
limit.  In your paper, please delete this instructions block and only keep the
Checklist section heading above along with the questions/answers below.
%%% END INSTRUCTIONS %%%

\begin{enumerate}

\item For all authors...
\begin{enumerate}
  \item Do the main claims made in the abstract and introduction accurately reflect the paper's contributions and scope?
    \answerTODO{}
  \item Did you describe the limitations of your work?
    \answerTODO{}
  \item Did you discuss any potential negative societal impacts of your work?
    \answerTODO{}
  \item Have you read the ethics review guidelines and ensured that your paper conforms to them?
    \answerTODO{}
\end{enumerate}

\item If you are including theoretical results...
\begin{enumerate}
  \item Did you state the full set of assumptions of all theoretical results?
    \answerTODO{}
	\item Did you include complete proofs of all theoretical results?
    \answerTODO{}
\end{enumerate}

\item If you ran experiments...
\begin{enumerate}
  \item Did you include the code, data, and instructions needed to reproduce the main experimental results (either in the supplemental material or as a URL)?
    \answerTODO{}
  \item Did you specify all the training details (e.g., data splits, hyperparameters, how they were chosen)?
    \answerTODO{}
	\item Did you report error bars (e.g., with respect to the random seed after running experiments multiple times)?
    \answerTODO{}
	\item Did you include the total amount of compute and the type of resources used (e.g., type of GPUs, internal cluster, or cloud provider)?
    \answerTODO{}
\end{enumerate}

\item If you are using existing assets (e.g., code, data, models) or curating/releasing new assets...
\begin{enumerate}
  \item If your work uses existing assets, did you cite the creators?
    \answerTODO{}
  \item Did you mention the license of the assets?
    \answerTODO{}
  \item Did you include any new assets either in the supplemental material or as a URL?
    \answerTODO{}
  \item Did you discuss whether and how consent was obtained from people whose data you're using/curating?
    \answerTODO{}
  \item Did you discuss whether the data you are using/curating contains personally identifiable information or offensive content?
    \answerTODO{}
\end{enumerate}

\item If you used crowdsourcing or conducted research with human subjects...
\begin{enumerate}
  \item Did you include the full text of instructions given to participants and screenshots, if applicable?
    \answerTODO{}
  \item Did you describe any potential participant risks, with links to Institutional Review Board (IRB) approvals, if applicable?
    \answerTODO{}
  \item Did you include the estimated hourly wage paid to participants and the total amount spent on participant compensation?
    \answerTODO{}
\end{enumerate}

\end{enumerate}

%%%%%%%%%%%%%%%%%%%%%%%%%%%%%%%%%%%%%%%%%%%%%%%%%%%%%%%%%%%%

\appendix

\section{Appendix}

Optionally include extra information (complete proofs, additional experiments and plots) in the appendix.
This section will often be part of the supplemental material.

\end{document}
