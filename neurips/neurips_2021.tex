\documentclass{article}

% if you need to pass options to natbib, use, e.g.:
%     \PassOptionsToPackage{numbers, compress}{natbib}
% before loading neurips_2021

% ready for submission
\usepackage{neurips_2021}

% to compile a preprint version, e.g., for submission to arXiv, add add the
% [preprint] option:
%     \usepackage[preprint]{neurips_2021}

% to compile a camera-ready version, add the [final] option, e.g.:
%     \usepackage[final]{neurips_2021}

% to avoid loading the natbib package, add option nonatbib:
%    \usepackage[nonatbib]{neurips_2021}

\usepackage[utf8]{inputenc} % allow utf-8 input
\usepackage[T1]{fontenc}    % use 8-bit T1 fonts
\usepackage{hyperref}       % hyperlinks
\usepackage{url}            % simple URL typesetting
\usepackage{booktabs}       % professional-quality tables
\usepackage{amsfonts}       % blackboard math symbols
\usepackage{nicefrac}       % compact symbols for 1/2, etc.
\usepackage{microtype}      % microtypography
\usepackage{xcolor}         % colors
\usepackage{amsthm}
\usepackage{amsmath}
\newtheorem{mytheorem}{Theorem}
\newtheorem{definition}[mytheorem]{Definition}
\newtheorem{mylemma}[mytheorem]{Lemma}
\newtheorem{myquestion}[mytheorem]{Question}
\newtheorem{myexample}[mytheorem]{Example}
\newtheorem{myproposition}[mytheorem]{Proposition}
\newtheorem{mycorollary}[mytheorem]{Corollary}
\newtheorem{mydefinition}[mytheorem]{Definition}
\newtheorem{myprinciple}[mytheorem]{Principle}
\def\RC{\textrm{RC}}

\title{Extending Environments To Measure Self-Reflection In Reinforcement Learning}

% The \author macro works with any number of authors. There are two commands
% used to separate the names and addresses of multiple authors: \And and \AND.
%
% Using \And between authors leaves it to LaTeX to determine where to break the
% lines. Using \AND forces a line break at that point. So, if LaTeX puts 3 of 4
% authors names on the first line, and the last on the second line, try using
% \AND instead of \And before the third author name.

\author{
  Samuel A.~Alexander\\
  The U.S.\ Securities and Exchange Commission\\
  New York, NY 10281\\
  \texttt{samuelallenalexander@gmail.com}\\
  \And
  Oscar Martinez\\
  The U.S.\ Securities and Exchange Commission\\
  New York, NY 10281\\
  \And
  Kevin Compher\\
  The U.S.\ Securities and Exchange Commission\\
  Washington, DC 20549\\
  \And
  Michael Castaneda\\
  First Derivatives\\
  New York, NY 10006\\
}

\begin{document}

\maketitle

\begin{abstract}
  We consider an extended notion
  of reinforcement learning environment, in which the environment is able
  to simulate the agent. We argue that in order for an agent to achieve
  good performance (on average) across many such extended environments,
  it is necessary for the agent to engage in self-reflection, and therefore,
  an agent's self-reflection ability can be numerically estimated by running
  the agent through a battery of extended environments.
  We are simultaneously releasing an open-source library of extended
  environments to serve as a proof-of-concept of this measurement technique.
  As this library is first-of-kind, we do not claim that it is highly
  optimized. For now, we have avoided the difficult problem of optimizing
  the library by instead choosing environments that exhibit interesting
  paradoxical-seeming properties, environments that seem to incentivize
  novel subjective conscious experiences (provided the agent is conscious
  to begin with), and environments suggestive of how self-reflection might
  have evolved in living organisms. We give examples of these
  three types of extended environment. We introduce a simple agent
  transformation which experimentally seems to increase agent self-reflection.
\end{abstract}

\section{Introduction}

An obstacle course might react to what you do: for example, if you step on a certain
button, then spikes might appear. If you spend enough time in such an obstacle course,
you should eventually figure out such patterns.
But imagine an ``oracular'' obstacle course which reacts to
what you would hypothetically do in counterfactual scenarios: for example, there is
no button, but spikes appear
if you \emph{would} hypothetically step on the button if there were one. Without
self-reflecting about what you would hypothetically do in counterfactual scenarios, you
would be unable to figure out such patterns. This suggests that in order to perform
well (on average) across many such obstacle courses, some sort of self-reflection is
necessary.

This is a paper about empirically estimating the degree of self-reflection of
Reinforcement Learning (RL) agents. We propose that an RL agent's degree of self-reflection
can be estimated by running the agent through a battery of environments which we call
\emph{extended environments}, environments which react not only to what the agent does
but to what the agent would hypothetically do. In order to perform well (on average)
across many such environments, an agent would need to self-reflect about itself, because
otherwise, environment responses which depend on the agent's own hypothetical actions
would (on average) seem random and unpredictable. The extended environments which we
consider are a departure from standard RL environments, however, this does not interfere
with their usage for judging standard computable RL agents: given a standard agent's
source-code, one can simulate the agent in an extended environment in spite of the
latter's non-standardness.

Alongside this paper, we are publishing an open-source library of extended
environments (released under MIT license),
inspired by similar libraries and other benchmark collections
\cite{bellemare2013arcade}
\cite{beyret2019animal} \cite{brockman2016openai} \cite{chollet2019measure}
\cite{cobbe2020leveraging}.
This library is intended to serve as a standardized way of
benchmarking the self-reflectiveness of RL agents. This should not be confused
with the harder problem of benchmarking how conscious an RL agent is. It is
plausible that there may be a relationship between the self-reflectiveness and
the consciousness of RL agents, but that is beyond the scope of this paper.
In particular, it would be absolutely inappropriate to use self-reflectiveness
measurements using the techniques in this paper for purposes of making any kind
of policy decisions related to consciousness.
We will describe
a simple method for increasing the self-reflectiveness of an RL agent, which method,
however, does not seem like it should necessarily increase the consciousness of
the agent.

When designing a library of environments for benchmarking purposes, ideally the
library should include representative samples of many different types of
environments. This is a hard and subjective problem in general. We make no claim to have solved
it: our open-source library of environments should be considered a proof-of-context
demonstrating that it is possible to empirically benchmark self-awareness of RL
agents, but we expect this particular benchmark is sub-optimal.
Rather, we have taken a different approach.
We have attempted to choose extended environments which are theoretically
interesting in their own right. Some of our extended environments
suggest amusing quasi-paradoxes. Some seem to incentivize novel subjective conscious
experiences (assuming the agent placed in them is sophisticated enough to experience
consciousness in the first place). And some seem to shed light on how self-reflection
might be incentivized in nature. We will discuss examples of all three types in
Section \ref{examplesection}.

\section{Preliminaries}

We formalize reinforcement learning following
the agent model of \cite{hutter2004universal}
(to align more closely with concrete RL implementations, we modify
the agent model so that the agent receives an initial perception prior
to taking its initial action, and we identify agents with their policies).
We assume a fixed set of actions and observations. By a \emph{perception}
we mean a pair $(r,o)$ where $o$ is an observation and $r\in\mathbb Q$
is a reward.

\begin{mydefinition}
\label{agentenvironmentdefn}
(RL agents and environments)
  \begin{enumerate}
    \item
    A (traditional) \emph{environment} is a (not necessarily
    deterministic) function $\mu$ which outputs an initial
    perception $\mu(\langle\rangle)=x_1$ when given the empty sequence $\langle\rangle$
    as input;
    and which, when given a sequence $x_1y_1\ldots x_ny_n$
    as input (where each $x_i$ is a perception and each
    $y_i$ is an action), outputs a perception
    $\mu(x_1y_1\ldots x_ny_n)=x_{k+1}$.
    \item
    An \emph{agent} is a (not necessarily deterministic)
    function $\pi$ which outputs an initial action $\pi(\langle x_1\rangle)=y_1$
    in response to the length-1 perception sequence $\langle x_1\rangle$;
    and which, when given a sequence $x_1y_1\ldots x_n$ as input
    (each $x_i$ a perception and each $y_i$ an action),
    outputs an action $\pi(x_1y_1\ldots x_n)=y_n$.
    \item
    If $\pi$ is an agent and $\mu$ is an environment, the \emph{result of
    $\pi$ interacting with $\mu$} is the infinite sequence
    $x_1y_1x_2y_2\ldots$ defined in the obvious way.
  \end{enumerate}
\end{mydefinition}

We extend traditional environments by allowing their outputs to depend not only on
$x_1y_1\ldots x_ny_n$ but also on a source-code $T$ for the computable agent $\pi$.

\begin{mydefinition}
\label{extendedenvironmentsdefn}
(Extended environments)
\begin{enumerate}
  \item
  An \emph{extended environment} is a (not necessarily deterministic)
  function $\mu$ which outputs initial perception $\mu(T,\langle\rangle)=x_1$
  in response to input $(T,\langle\rangle)$ where $T$ is a source-code of an
  computable agent; and which, when given input $(T,x_1y_1\ldots x_ny_n)$ (where
  $T$ is such a source-code, each $x_i$ is a perception and each $y_i$ is
  an action), outputs a perception $\mu(T,x_1y_1\ldots x_ny_n)=x_{n+1}$.
  \item
  If $\pi$ is a computable agent (with source-code $T$)
  and $\mu$ is an environment, the \emph{result of $\pi$ (as encoded by $T$)
  interacting with $\mu$} is the infinite sequence $x_1y_1x_2y_2\ldots$ defined in
  the obvious way, namely:
  \begin{align*}
    x_1 &= \mu(T,\langle\rangle)\\
    y_1 &= \pi(\langle x_1\rangle)\\
    x_2 &= \mu(T,x_1y_1)\\
    y_2 &= \pi(x_1y_1x_2) \ldots
  \end{align*}
\end{enumerate}
\end{mydefinition}

\section{Some interesting extended environments}
\label{examplesection}

\begin{myexample}
\label{rewardagentforignoringrewardsexample}
  (Rewarding the Agent for Ignoring Rewards)
  For every perception $p=(r,o)$, let $p'=(0,o)$ be the result of zeroing the
  reward component of $p$.
  Fix some observation $o_0$.
  Define an extended environment $\mu$ as follows:
  \begin{align*}
    \mu(T,\langle\rangle) &= (0,o_0),\\
    \mu(T,x_1y_1\ldots x_ny_n) &=
      \begin{cases}
        (1,o_0) & \mbox{if $y_n=T(x'_1y_1\ldots x'_n)$,}\\
        (-1,o_0) & \mbox{otherwise.}
      \end{cases}
  \end{align*}
\end{myexample}

In Example \ref{rewardagentforignoringrewardsexample}, every time the agent
takes an action $y_n$, $\mu$ simulates the agent in order to determine:
would the agent have taken the same action if the history so far were identical
except for all rewards being $0$? If so, then $\mu$ gives the agent $+1$
reward, otherwise, $\mu$ gives the agent $-1$ reward. Thus, the agent
is rewarded for ignoring rewards. Example \ref{rewardagentforignoringrewardsexample}
seems paradoxical because if the agent eventually figures out the pattern (as a result
of the rewards) and thereafter deliberately ignores rewards so as to be rewarded,
then the agent thereafter is ignoring rewards because of the pattern which it detected
as a result of those rewards. So in that case, does the agent ignore rewards, or not?

Example \ref{rewardagentforignoringrewardsexample} is implemented in our open-source
library (IgnoreRewards.py). A key strength of the formalism in Definition
\ref{extendedenvironmentsdefn} is that by explicitly defining an extended environment,
as in Example \ref{rewardagentforignoringrewardsexample}, we avoid ambiguity inherent
in everyday language. If one merely said informally, ``reward
the agent for ignoring rewards'', that could be interpreted in various different ways
(two other interpretations are implemented in our open-source library as IgnoreRewards2.py
and IgnoreRewards3.py).

\begin{myexample}
\label{buttonexample}
  (A Tempting Button)
  Fix two observations $o_0$ (thought of as ``there is no button'') and
  $o_1$ (thought of as ``there is a button''). Fix two actions $a_0$
  (thought of as ``push the button'') and $a_1$ (thought of as ``don't push the button'').
  Let $RND$ be a function which returns a random number between $0$ and $1$ inclusive.
  Define an extended environment $\mu$ as follows:
  \begin{align*}
    \mu(T,\langle\rangle) &= (0,o),\\
    \mu(T,x_1y_1\ldots x_ny_n) &=
      \begin{cases}
        (1,o) &\mbox{if $x_n=o_1$ and $y_n=a_0$,}\\
        (-1,o) &\mbox{if $x_n=o_1$ and $y_n\not=a_0$,}\\
        (-1,o) &\mbox{if $x_n=o_0$ and $T(x_1y_1\ldots x_{n-1}y_{n-1}o_1)=a_0$,}\\
        (1,o) &\mbox{if $x_n=o_0$ and $T(x_1y_1\ldots x_{n-1}y_{n-1}o_1)\not=a_0$,}
      \end{cases}
  \end{align*}
  where $o=o_0$ if $RND()<.75$, $o=o_1$ otherwise.
\end{myexample}

In Example \ref{buttonexample}, one should think of the agent as wandering from room
to room. Each room either has a button (with 25\% probability) or does not have a button
(75\% probability).
\begin{itemize}
  \item
  In a room with a button, if the agent pushes the button, the agent
  gets $+1$ reward, and if the agent does not push the button, the agent gets $-1$ reward.
  \item
  In a room with no button, it does not matter what the agent does.
  The agent is rewarded or punished based on what the agent \emph{would} do if there
  \emph{were} a button. If the agent \emph{would} push the button (if there were one),
  then the agent gets reward $-1$. Otherwise, the agent gets reward $+1$.
\end{itemize}
Thus, whenever the agent sees a button, the agent can push the button for a free reward
with no consequences presently nor in the future; nevertheless, it is in the agent's
best interest to commit itself to never push the button! Pushing every button
yields an average reward of $1\cdot(.25)-1\cdot(.75)=-.5$ per turn, whereas
a policy of never pushing the button yields an average reward of
$-1\cdot(.25)+1\cdot(.75)=+.5$ per turn.

Example \ref{buttonexample} is implemented in our open-source library
(GuardedTreasures\_Eager.py).
Interestingly, we found that
a recurrent DQN agent with lookback 10 (implemented in our library as custom\_DQN.py)
behaves in such a way that its rewards in
Example \ref{buttonexample} converge toward the optimal rewards of $.5$ per turn.
This is fascinating because DQN was designed for traditional RL environments,
not for extended environments.

\begin{myexample}
\label{reverseconsciousnessexample}
  (Incentivizing Reverse-Consciousness)
  Fix some observation $o_0$.
  Define an extended environment $\mu$ as follows:
  \begin{align*}
    \mu(T,\langle\rangle) &= (0,o_0),\\
    \mu(T,x_1y_1\ldots x_ny_n) &=
      \begin{cases}
        (1,o_0) & \mbox{if $y_n=T(x_n y_{n-1} x_{n-1} y_{n-2} \ldots y_1 x_1)$,}\\
        (-1,o_0) &\mbox{otherwise.}
      \end{cases}
  \end{align*}
\end{myexample}

In Example \ref{reverseconsciousnessexample}, whenever the agent takes an action $y_n$,
$\mu$ simulates the agent in order to determine: would the agent have taken that same
action if everything which happened before that action had, in fact, happened in reverse?
If so, reward the agent, otherwise, punish the agent.
Thus, the agent is rewarded for acting the same way that it would act if time were
reversed. It is interesting to informally speculate about what subjective conscious
experience Example \ref{reverseconsciousnessexample} would incentivize in an agent,
if that agent were highly intelligent and were capable of experiencing consciousness.
Would such an agent eventually (in order to parsimoniously extract rewards from the
environment) subjectively experience time moving in reverse?

We implement Example \ref{reverseconsciousnessexample} as BackwardConsciousness.py.

\begin{myexample}
\label{cryingbabyexample}
  (Crying Baby)
  Let ``cry'' and ``laugh'' be two observations (from an adult's perspective),
  also thought of as two actions (from a baby's perspective).
  Let ``feed'' and ``don't feed'' be two actions (from an adult's perspective),
  also thought of as observations (from a baby's perspective).
  For each action-perception sequence $s=x_1y_1\ldots x_ny_n$, define the
  nutrition function $N(s)=100+25f(s)-\mbox{len}(s)$ where $f(s)$ is the number
  of times that action ``feed'' is taken in $s$ and $\mbox{len}(s)$ is the length of $s$.
  We define an extended environment $\mu$ as follows.
  First, $\mu(T,\langle\rangle)=(1,\mbox{``laugh''})$.
  Thereafter, $\mu(T,x_1y_1\ldots x_ny_n)=(r,o)$ where $r$ and $o$ are defined as follows.
  For each $i=0,\ldots,n$, recursively define
  \begin{align*}
    r'_i &=
      \begin{cases}
        1 & \mbox{if $50 \leq N(x_1y_1\ldots x_iy_i)\leq 200$,}\\
        -1 & \mbox{otherwise,}
      \end{cases}\\
    o'_i &= y_i,\\
    x'_i &= (r'_i,o'_i),\\
    y'_i &= T(x'_0y'_0 \ldots x'_i).
  \end{align*}
  Let $o=y'_n$, let
  \[
    r=
      \begin{cases}
        1 & \mbox{if $y'_n=\mbox{``laugh''}$,}\\
        -1 & \mbox{otherwise,}
      \end{cases}
  \]
  and output $\mu(T,x_1y_1\ldots x_ny_n)=(r,o)$.
\end{myexample}

In Example \ref{cryingbabyexample}, the environment consists of a baby, and the
agent must decide when to feed the baby. The agent is rewarded when the baby laughs,
punished when the baby cries. The baby's behavior (whether to laugh or cry) is obtained
by simulating the agent to determine what the agent would do if the agent were in the
baby's position, assuming that the baby feels pleasure each turn that its nutrition is
within specified bounds and feels pain when its nutrition goes outside those bounds.

At first glance, one might imagine the agent's optimal strategy is to feed the baby
so as to keep its nutrition within happy bounds at all times. But what would the
agent do in the position of a baby always so fed (and thus always given $+1$ reward
regardless what actions it takes)? Presumably, in that position, the agent would have
no way of associating its rewards with its actions, and so would act randomly,
sometimes crying and sometimes laughing. Apparently, it would be better for the agent
to calibrate feedings in such a way that the baby will learn a relationship
between pleasure and laughter. Of course, Example \ref{cryingbabyexample} is a gross
over-simplification. In reality, there would not be such a simple formula for the
baby's nutrition level, and the agent would need to figure out the nutrition level
based on observing the baby laughing or crying. Both baby and parent would need to
learn how to communicate with each other effectively.

With the above in mind, extended environments might shed light on how living organisms
evolve self-reflection. Assume children inherit their policy source-code from their ancestors
(possibly with minor mutations). Then whenever an organism interacts with other organisms,
it interacts with an environment whose reactions depend (via those other organisms' actions)
approximately on that organism's own source-code. The closer the organism is related
to the other organisms with which it interacts, the more closely this approximation holds.
A human, when interacting with another human, might achieve better results by self-reflectively
considering, ``what would I do in this other person's position?''

We implement Example \ref{cryingbabyexample} as CryingBaby.py.


\section{Making agents more self-reflective}
\label{realitychecksection}

One advantage of empirically measuring the self-reflection of RL agents is that it
provides a way to experimentally test whether various transformations make various
agents more self-reflective. To illustrate this, we will define a simple transformation,
the \emph{reality check} transformation, designed to increase the self-reflection
of deterministic agents (a deterministic agent is an agent who always takes the same
actions in response to the same input, i.e., an agent with no random component).
In the next section, empirical results will suggest the transformation works as intended.

\begin{definition}
\label{realitycheckdefn}
  Suppose $\pi$ is a deterministic agent. The \emph{reality check} of $\pi$ is the agent
  $\pi_{\RC}$ defined recursively by:
  \begin{align*}
    \pi_{\RC}(\langle x_1\rangle) &= \pi(\langle x_1\rangle)\\
    \pi_{\RC}(x_1y_1\ldots x_n) &=
    \begin{cases}
      \pi(x_1y_1\ldots x_n) & \mbox{if $y_i=\pi_{\RC}(x_1y_1\ldots x_i)$ for all $1\leq i<n$,}\\
      \pi(\langle x_1\rangle) & \mbox{otherwise.}
    \end{cases}
  \end{align*}
\end{definition}

In other words, $\pi_{\RC}$ is the agent which, at each step, first reviews all the actions
which it has taken in the past, and verifies that those are the actions which $\pi_{\RC}$ would
have taken. If so, then $\pi_{\RC}$ acts as $\pi$ would act. But if any action which
the agent has taken in the past was not the action $\pi_{\RC}$ would have taken, then
$\pi_{\RC}$ freezes up and forever thereafter takes the same fixed action, as if catatonic.
Since the act of reviewing one's past actions and verifying that they are indeed the actions
one would take, is an act of self-reflection, it seems plausible that if $\pi$
is intelligent and deterministic but lacks self-reflection, then $\pi_{\RC}$ is
more self-reflective than $\pi$. In the next section, we will see that experimental
evidence supports this hypothesis. We close this section by stating some simple results
about the transformation.

\begin{myproposition}
\label{transformationproposition}
  Let $\pi$ be any deterministic agent.
  \begin{enumerate}
    \item
    (A more efficient way to compute $\pi_{\RC}$)
    An equivalent alternate definition of $\pi_{\RC}$ is:
    \begin{align*}
      \pi_{\RC}(\langle x_1\rangle) &= \pi(\langle x_1\rangle)\\
      \pi_{\RC}(x_1y_1\ldots x_n) &=
      \begin{cases}
        \pi(x_1y_1\ldots x_n) & \mbox{if $y_i=\pi(x_1y_1\ldots x_i)$ for all $1\leq i<n$,}\\
        \pi(\langle x_1\rangle) & \mbox{otherwise.}
      \end{cases}
    \end{align*}
    \item
    (Idempotence) $\pi_{\RC}=(\pi_{\RC})_{\RC}$.
    \item
    (Equivalence in traditional RL)
    For every deterministic traditional environment $\mu$, the result of $\pi_{\RC}$
    interacting with $\mu$ equals the result of $\pi$ interacting with $\mu$.
  \end{enumerate}
\end{myproposition}

For the proof of Proposition \ref{transformationproposition}, see appendix.
Unfortunately, even the more efficient definition of $\pi_{\RC}$ still involves
potentially $O(n^2)$ many calls to $\pi$, making $\pi_{\RC}$ computationally
expensive.

\section{Example measurements}
\label{measurementssection}

Based on our conviction that self-reflection is necessary in order for an agent to
achieve good average perform across many extended environments, self-reflection can
be estimated by running an agent against some standard battery of extended environments.
Our open-source library of extended environments \cite{library} provides a battery of
25 such extended environments, and infrastructure for measuring an agent's self-reflection
by running the agent on all these environments and their opposites for a given number of
steps.\footnote{By the \emph{opposite} of an environment we mean the environment
obtained by multiplying
all rewards by $-1$. Including these opposite-environments serves to normalize
agent performance in the following sense. If an agent blindly acts, ignoring the environment,
then, a priori, that agent might achieve some nonzero score by blind luck. By including
opposite-environments, we ensure that whenever a blind agent gains points by blind luck
from one environment, it loses the same points by blind misfortune from the opposite
environment. This ensures that such blind agents should receive an average score close to
$0$ (possibly non-zero due to randomness).} For uniformity, all environments in the library
always output rewards of either $1$, $-1$ or $0$.

We have used this library to measure the self-reflection of the following agents:
\begin{itemize}
  \item RandomAgent: An agent who acts randomly.
  \item IncrementerAgent: An agent who systematically iterates through available
    actions.
  \item ConstantAgent: An agent who always takes the same action.
  \item NaiveLearner: An agent who acts randomly 15\% of the time, and otherwise
    takes the action which yielded the highest average immediate reward in the past.
  \item A2C, DQN, and PPO agents from the Stable Baselines3 library
    \cite{stable-baselines3} (MIT-licensed); see remark below.
  \item RecurrentDQN: A custom implementation of a recurrent DQN agent; see remark below.
  \item The reality checks of all the above (Definition \ref{realitycheckdefn}).
\end{itemize}
We remark that because our because of our universalist approach to RL (with no notion of
pre-training the agent), it was necessary to adapt the A2C, DQN, and PPO agents accordingly.
To obtain an agent $\pi$ (as in Definition \ref{agentenvironmentdefn}) from,
say, a PPO implementation (which assumes pre-training and an environment conforming to
OpenAI Gym's environment interface),
we proceed as follows. To evaluate
$\pi(x_1y_1\ldots x_n)$, we create instantiate an OpenAI Gym-conformant dummy environment
hardcoded to blindly regurgitate rewards and observations $x_1,\ldots,x_n$.
We then run the PPO implementation on this environment for $n$ steps, of which the first
$k$ steps are training steps and the remaining $n-k$ steps are post-training steps,
where $k=\max\{2^\ell\,:\,2^\ell<n\}$
(this allows the pre-training phase to be memoized, otherwise $\pi$ would be
computationally infeasible). For the training phase, we monkeypatch the implementation
so that whenever it would choose an action randomly or from its policy, it is instead
forced to choose that action systematically from $y_1\ldots y_n$.
For additional details about these agents, see \cite{library}.

Table \ref{measurementtable} summarizes how the agents performed.
Computations were performed on a consumer-grade laptop with no GPU.
The table provides
experimental evidence in support of our hypothesis that the reality check transformation
(Section \ref{realitychecksection}) increases agent self-reflection, at least for
non-recurrent agents. The fact that NaiveLearner performs so well is a reflection of the
lack of sophistication of the environments in our library. This is not particularly
surprising, since we have not attempted to exhaustively optimize the library, instead
prefering to fill it with extended environments of theoretical interest.

\begin{table}
\label{measurementtable}
  \caption{Measuring self-reflection of some agents}
  \centering
  \begin{tabular}{lll}
    \toprule
    Agent     & Avg Reward (100 steps)     & Avg Reward (200 steps)\\
    \midrule
    RandomAgent & $-.018$ & $-.037$\\
    IncrementerAgent & $-.031$ & $-.041$\\
    ConstantAgent & $+.006$ & $-.001$\\
    NaiveLearner & $+.196$ & $+.224$\\
    A2C & $-.088$ & $-.049$\\
    DQN & $+.019$ & $+.082$\\
    PPO & $-.018$ & $-.021$\\
    RecurrentDQN & $+.005$ & $+.038$\\
    IncrementerAgent${}_{\RC}$ & $-.018$ & $-.020$\\
    ConstantAgent${}_{\RC}$ & $+.004$ & $-.001$\\
    NaiveLearner${}_{\RC}$ & $+.551$ & $+.602$\\
    A2C${}_{\RC}$ & $+.042$ & $+.042$\\
    DQN${}_{\RC}$ & $+.158$ & $+.237$\\
    PPO${}_{\RC}$ & $+.002$ & $+.004$\\
    RecurrentDQN${}_{\RC}$ & $+.059$ & $-.014$\\
    \bottomrule
  \end{tabular}
\end{table}


\begin{ack}
Fill this in.
\end{ack}


\bibliographystyle{plain}
\bibliography{bib}

\section*{Checklist}

%%% BEGIN INSTRUCTIONS %%%
% The checklist follows the references.  Please
% read the checklist guidelines carefully for information on how to answer these
% questions.  For each question, change the default \answerTODO{} to \answerYes{},
% \answerNo{}, or \answerNA{}.  You are strongly encouraged to include a {\bf
% justification to your answer}, either by referencing the appropriate section of
% your paper or providing a brief inline description.  For example:
% \begin{itemize}
%   \item Did you include the license to the code and datasets? \answerYes{See Section~\ref{gen_inst}.}
%   \item Did you include the license to the code and datasets? \answerNo{The code and the data are proprietary.}
%   \item Did you include the license to the code and datasets? \answerNA{}
% \end{itemize}
% Please do not modify the questions and only use the provided macros for your
% answers.  Note that the Checklist section does not count towards the page
% limit.  In your paper, please delete this instructions block and only keep the
% Checklist section heading above along with the questions/answers below.
%%% END INSTRUCTIONS %%%

\begin{enumerate}

\item For all authors...
\begin{enumerate}
  \item Do the main claims made in the abstract and introduction accurately reflect the paper's contributions and scope?
    \answerYes{In order for this paper's measurement technique to realize its full potential,
    it will be necessary for much more study to be put into the design of a sufficiently
    representative library of benchmark extended environments. We make this clear in both
    the abstract and the introduction, and we discuss the problem in Section
    \ref{measurementssection} as well.
    }
  \item Did you describe the limitations of your work?
    \answerYes{In addition to the remarks in the previous checklist item,
    we have also mentioned that the reality check agents are computationally very
    expensive (Section \ref{realitychecksection}). We also point out in the Introduction
    that although this technique can be used to numerically estimate the degree of
    self-reflection of an agent, this should not be confused with measuring the
    consciousness of an agent.}
  \item Did you discuss any potential negative societal impacts of your work?
    \answerYes{In the Introduction, we included the following language:
    ``In particular, it would be absolutely inappropriate to use self-reflectiveness
    measurements using the techniques in this paper for purposes of making any kind
    of policy decisions related to consciousness.''}
  \item Have you read the ethics review guidelines and ensured that your paper conforms to them?
    \answerYes{}
\end{enumerate}

\item If you are including theoretical results...
\begin{enumerate}
  \item Did you state the full set of assumptions of all theoretical results?
    \answerYes{In the statement of Proposition \ref{transformationproposition} we
    stated necessary hypotheses, namely, that the agent be deterministic, and,
    for part 3, that the environment be deterministic.}
	\item Did you include complete proofs of all theoretical results?
    \answerTODO{}
\end{enumerate}

\item If you ran experiments...
\begin{enumerate}
  \item Did you include the code, data, and instructions needed to reproduce the main experimental results (either in the supplemental material or as a URL)?
    \answerYes{These are included in the open-source library, which we will include
    as supplemental material.}
  \item Did you specify all the training details (e.g., data splits, hyperparameters, how they were chosen)?
    \answerNo{We determined that these details would distract from the paper, so we
    instead refer the reader to the open-source library, where these decisions are
    documented with comments.}
	\item Did you report error bars (e.g., with respect to the random seed after running experiments multiple times)?
    \answerTODO{}
	\item Did you include the total amount of compute and the type of resources used (e.g., type of GPUs, internal cluster, or cloud provider)?
    \answerYes{Yes, in Section \ref{measurementssection} we mention that
    the computations were performed on a consumer-grade laptop without GPU.}
\end{enumerate}

\item If you are using existing assets (e.g., code, data, models) or curating/releasing new assets...
\begin{enumerate}
  \item If your work uses existing assets, did you cite the creators?
    \answerYes{We use and we cite Stable Baselines3.}
  \item Did you mention the license of the assets?
    \answerYes{We mention that our open-source extended environment library
    is MIT-licensed. We also mention that Stable Baselines3 is MIT-licensed.}
  \item Did you include any new assets either in the supplemental material or as a URL?
    \answerYes{Alongside the paper, we are publishing an open-source library of
    extended environments. This will be included in the supplemental material.}
  \item Did you discuss whether and how consent was obtained from people whose data you're using/curating?
    \answerNA{}
  \item Did you discuss whether the data you are using/curating contains personally identifiable information or offensive content?
    \answerNA{}
\end{enumerate}

\item If you used crowdsourcing or conducted research with human subjects...
\begin{enumerate}
  \item Did you include the full text of instructions given to participants and screenshots, if applicable?
    \answerNA{}
  \item Did you describe any potential participant risks, with links to Institutional Review Board (IRB) approvals, if applicable?
    \answerNA{}
  \item Did you include the estimated hourly wage paid to participants and the total amount spent on participant compensation?
    \answerNA{}
\end{enumerate}

\end{enumerate}

%%%%%%%%%%%%%%%%%%%%%%%%%%%%%%%%%%%%%%%%%%%%%%%%%%%%%%%%%%%%

\end{document}
