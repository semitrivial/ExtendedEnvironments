\documentclass{article}

% if you need to pass options to natbib, use, e.g.:
%     \PassOptionsToPackage{numbers, compress}{natbib}
% before loading neurips_2021

% ready for submission
\usepackage{neurips_2021}

% to compile a preprint version, e.g., for submission to arXiv, add add the
% [preprint] option:
%     \usepackage[preprint]{neurips_2021}

% to compile a camera-ready version, add the [final] option, e.g.:
%     \usepackage[final]{neurips_2021}

% to avoid loading the natbib package, add option nonatbib:
%    \usepackage[nonatbib]{neurips_2021}

\usepackage[utf8]{inputenc} % allow utf-8 input
\usepackage[T1]{fontenc}    % use 8-bit T1 fonts
\usepackage{hyperref}       % hyperlinks
\usepackage{url}            % simple URL typesetting
\usepackage{booktabs}       % professional-quality tables
\usepackage{amsfonts}       % blackboard math symbols
\usepackage{nicefrac}       % compact symbols for 1/2, etc.
\usepackage{microtype}      % microtypography
\usepackage{xcolor}         % colors
\usepackage{amsthm}
\usepackage{amsmath}
\newtheorem{mytheorem}{Theorem}
\newtheorem{definition}[mytheorem]{Definition}
\newtheorem{myremark}[mytheorem]{Remark}
\newtheorem{mylemma}[mytheorem]{Lemma}
\newtheorem{myquestion}[mytheorem]{Question}
\newtheorem{myexample}[mytheorem]{Example}
\newtheorem{myproposition}[mytheorem]{Proposition}
\newtheorem{mycorollary}[mytheorem]{Corollary}
\newtheorem{mydefinition}[mytheorem]{Definition}
\newtheorem{myprinciple}[mytheorem]{Principle}
\def\RC{\textrm{RC}}
\def\A{\mathcal{A}}
\def\O{\mathcal{O}}
\def\RND{\textrm{RND}}
\def\obs{\textrm{obs}}
\def\rwd{\textrm{rwd}}

\title{Extending Environments To Measure Self-Reflection In Reinforcement Learning}

% The \author macro works with any number of authors. There are two commands
% used to separate the names and addresses of multiple authors: \And and \AND.
%
% Using \And between authors leaves it to LaTeX to determine where to break the
% lines. Using \AND forces a line break at that point. So, if LaTeX puts 3 of 4
% authors names on the first line, and the last on the second line, try using
% \AND instead of \And before the third author name.

\author{
  Samuel A.~Alexander\\
  The U.S.\ Securities and Exchange Commission\\
  New York, NY 10281\\
  \texttt{samuelallenalexander@gmail.com}\\
  \And
  Oscar Martinez\\
  The U.S.\ Securities and Exchange Commission\\
  New York, NY 10281\\
  \And
  Kevin Compher\\
  The U.S.\ Securities and Exchange Commission\\
  Washington, DC 20549\\
  \And
  Michael Castaneda\\
  First Derivatives\\
  New York, NY 10006\\
}

\begin{document}

\maketitle

\begin{abstract}
  We consider an extended notion
  of reinforcement learning in which environments are able
  to simulate the agent. We argue that in order for an agent to achieve
  on average good performance across many such extended environments,
  it is necessary for the agent to engage in self-reflection, and therefore,
  an agent's self-reflection ability can be numerically estimated by running
  the agent through a battery of extended environments.
  We are simultaneously releasing an open-source library of extended
  environments to serve as a proof-of-concept of this measurement technique.
  As this library is first-of-kind, we do not claim that it is highly
  optimized and have avoided the difficult problem of optimizing it for now.
  Instead we have chosen environments that exhibit interesting, sometimes
  paradoxical-seeming properties, or that incentivize
  novel subjective conscious experiences, provided the agent is conscious
  to begin with. Some of these extended environments
  are suggestive of how self-reflection might have evolved in living organisms.
  We give examples of extended environments and introduce a simple agent
  transformation which experimentally seems to increase agent self-reflection.
\end{abstract}

\section{Introduction}

An obstacle course might react to what you do: for example, if you step on a certain
button, then spikes might appear. If you spend enough time in such an obstacle course,
you should eventually figure out such patterns.
But imagine an ``oracular'' obstacle course which reacts to
what you would hypothetically do in counterfactual scenarios: for example, there is
no button, but spikes appear
if you \emph{would} hypothetically step on the button if there was one. Without
self-reflecting about what you would hypothetically do in counterfactual scenarios, it
would be difficult to figure out such patterns. This suggests that in order to perform
well (on average) across many such obstacle courses, some sort of self-reflection is
necessary.

This is a paper about empirically estimating the degree of self-reflection of
Reinforcement Learning (RL) agents. We propose that an RL agent's degree of self-reflection
can be estimated by running the agent through a battery of environments which we call
\emph{extended environments}, environments which react not only to what the agent does
but to what the agent would hypothetically do.
For good performance averaged over
many such environments, an agent would need to self-reflect about itself, because
otherwise, environment responses which depend on the agent's own hypothetical actions
would often seem random and unpredictable. The extended environments which we
consider are a departure from standard RL environments, however, this does not interfere
with their usage for judging standard computable RL agents: given a standard agent's
source-code, one can simulate the agent in an extended environment in spite of the
latter's non-standardness.

One might try to imitate an extended environment with a non-extended environment by
backtracking---rewinding the environment itself to a prior state after seeing how the
agent performs along one path, and then sending the agent along a second path.
But the agent itself would retain memory of the first path, and the agent's decisions
along the second path might be altered by said memories. Thus the result would not be
the same as immediately sending the agent along the second path while secretly simulating
the agent to determine what it would do if sent along the first path.

Alongside this paper, we are publishing an open-source library \cite{library} of extended
environments (released under MIT license)
to ``ease adoption by other machine-learning researchers''
\cite{sonnenburg2007need}.
We are
inspired by similar libraries and other benchmark collections
\cite{bellemare2013arcade}
\cite{beyret2019animal} \cite{brockman2016openai} \cite{chollet2019measure}
\cite{cobbe2020leveraging}
\cite{hendrycks2019benchmarking}
\cite{yampolskiy2017detecting}.
This library is intended to serve as a standardized way of
benchmarking the self-reflectiveness of RL agents. This should not be confused
with the harder problem of benchmarking how conscious an RL agent is. It is
plausible that there may be a relationship between the self-reflectiveness and
the consciousness of RL agents, but that is beyond the scope of this paper.
In particular, it would be inappropriate to use self-reflectiveness
measurements from this paper for purposes of making any kind
of policy-decisions related to consciousness.
We will describe (in Section \ref{realitychecksection})
a simple method for increasing the self-reflectiveness of an RL agent, which method,
however, does not seem like it should necessarily increase the consciousness of
the agent.

When designing a library of environments for benchmarking purposes, ideally the
library should include properly weighted representative samples of many types of
environments. This is a hard and subjective problem in general
\cite{leike2015bad}. We make no claim to have solved
it: our open-source library of environments should be considered a proof-of-concept
demonstrating that it is possible to empirically benchmark self-awareness of RL
agents, but we expect this particular benchmark is sub-optimal.
Rather, we have taken a different approach.
We have attempted to choose extended environments which are theoretically
interesting in their own right. Some of our extended environments
suggest amusing quasi-paradoxes (somewhat like Newcomb's paradox
\cite{nozick1969newcomb}). Some seem to incentivize novel subjective conscious
experiences (assuming the agent placed in them is sophisticated enough to experience
consciousness in the first place). And some seem to shed light on how self-reflection
might be incentivized in nature. We will discuss examples of all three types in
Section \ref{examplesection}.

\section{Preliminaries}

We formalize reinforcement learning following
the agent model of \cite{hutter2004universal},
except that,
to align more closely with concrete RL implementations,
we assume the agent receives an initial percept before
taking its initial action.

Our formalization differs from how RL agents are
implemented in practice. In Section \ref{mappingtouniversalsection}
we will discuss how practical RL agent implementations can be
transformed into agents of this abstract type.

We assume fixed finite sets of actions and observations. By a \emph{percept}
we mean a pair $(r,o)$ where $o$ is an observation and $r\in\mathbb Q$
is a reward.

\begin{mydefinition}
\label{agentenvironmentdefn}
(RL agents and environments)
  \begin{enumerate}
    \item
    A (non-extended) \emph{environment} is a (not necessarily
    deterministic) function $\mu$ which outputs an initial
    percept $\mu(\langle\rangle)=x_1$ when given the empty sequence $\langle\rangle$
    as input
    and which, when given a sequence $x_1y_1\ldots x_ny_n$
    as input (where each $x_i$ is a percept and each
    $y_i$ is an action), outputs a percept
    $\mu(x_1y_1\ldots x_ny_n)=x_{n+1}$.
    \item
    An \emph{agent} is a (not necessarily deterministic)
    function $\pi$ which outputs an initial action $\pi(\langle x_1\rangle)=y_1$
    in response to the length-1 percept sequence $\langle x_1\rangle$;
    and which, when given a sequence $x_1y_1\ldots x_n$ as input
    (each $x_i$ a percept and each $y_i$ an action),
    outputs an action $\pi(x_1y_1\ldots x_n)=y_n$.
    \item
    If $\pi$ is an agent and $\mu$ is an environment, the \emph{result of
    $\pi$ interacting with $\mu$} is the infinite sequence
    $x_1y_1x_2y_2\ldots$ defined in the obvious way.
  \end{enumerate}
\end{mydefinition}

We extend environments by allowing their outputs to depend not only on
$x_1y_1\ldots x_ny_n$ but also on a source-code $T$ for the computable agent $\pi$.

\begin{mydefinition}
\label{extendedenvironmentsdefn}
(Extended environments)
\begin{enumerate}
  \item
  An \emph{extended environment} is a (not necessarily deterministic)
  function $\mu$ which outputs initial percept $\mu(T,\langle\rangle)=x_1$
  in response to input $(T,\langle\rangle)$ where $T$ is a source-code of a
  computable agent; and which, when given input $(T,x_1y_1\ldots x_ny_n)$ (where
  $T$ is such a source-code, each $x_i$ is a percept and each $y_i$ is
  an action), outputs a percept $\mu(T,x_1y_1\ldots x_ny_n)=x_{n+1}$.
  \item
  If $\pi$ is a computable agent (with source-code $T$)
  and $\mu$ is an environment, the \emph{result of $\pi$ (as encoded by $T$)
  interacting with $\mu$} is the infinite sequence $x_1y_1x_2y_2\ldots$ defined in
  the obvious way, namely:
  \begin{align*}
    x_1 &= \mu(T,\langle\rangle)\\
    y_1 &= \pi(\langle x_1\rangle)\\
    x_2 &= \mu(T,x_1y_1)\\
    y_2 &= \pi(x_1y_1x_2) \ldots
  \end{align*}
\end{enumerate}
\end{mydefinition}

The fact that agents can interact normally with extended environments
(Definition \ref{extendedenvironmentsdefn} part 2) implies that various universal
RL intelligence measures
\cite{goertzel2006patterns} \cite{hernandez} \cite{gavane} \cite{legg2007universal}
have straightforward analogous measures which also take
extended environments into account and which might therefore measure some combination
of intelligence and self-reflection.

\begin{myremark}
  In the accompanying extended environment library \cite{library}, we formalize agents
  and extended environments slightly differently than above.
  For better interoperability
  with practical RL implementations, instead of fixing finite
  sets of actions and observations globally, each of the library's extended environments
  specifies how many actions and observations are legal in that environment.
  These numbers are passed to agents as additional inputs.
  We have simplified the formalization in the paper because it simplifies
  mathematical notation while, we believe, not hiding any insights.
\end{myremark}

\subsection{Converting practical RL agents to agents as in
Definition \ref{agentenvironmentdefn}}
\label{mappingtouniversalsection}

The agents in Definition \ref{agentenvironmentdefn} have no training phase. They must be
ready to perform in any environment, out-of-the-box. Practical RL agent implementations,
on the other hand, are designed with the assumption that the user is interested in one
particular environment (say, an environment compliant with OpenAI Gym's
environment interface)---``a single function in isolation'' \cite{thrun1998lifelong}.
A typical practical RL agent takes one single observation (or
\emph{state}) as input, rather than a percept-action sequence, and its output is based
on certain \emph{weights} (say, neural network weights). If the agent is in the midst of
\emph{training}, then, after it acts, its weights might be updated based on the environment's
response. If the agent is not in the midst of training, then its weights remain fixed.

Given a typical practical RL agent as described above, one simple
(but too slow, see below)
way to obtain an agent
$\pi$ as in Definition \ref{agentenvironmentdefn} is as follows.
Given an input $x_1y_1\ldots x_n$ (where each $x_i$ is a percept and each $y_i$ is an
action), compute $\pi(x_1y_1\ldots x_n)$ as follows. First, instantiate a dummy
environment (say, an environment compliant with OpenAI Gym's interface, or whatever other
interface the practical RL agent expects) which is hardcoded to blindly regurgitate
$x_1,\ldots,x_n$ as its first $n$ responses, regardless of what the agent does.
Then train the agent on this environment for $n$ steps, with instructions to
take $y_1,\ldots,y_{n-1}$ as its first $n-1$ actions, and finally, let
$\pi(x_1y_1\ldots x_n)$ be whatever $n$th action the agent chooses as a result.
Unfortunately, in our experience, the typical practical RL agent probably does not have
any way for the user to tell it to take $y_1,\ldots,y_{n-1}$ as its first $n-1$
actions. Instead, if left to itself, the agent would choose its first $n-1$ actions
randomly, or based on an underlying policy (usually still with an element of randomness).
Fortunately, most practical RL agents are implemented in Python, and can therefore
be monkeypatched in order to override the randomness of the action choice and ensure
that the random number generator chooses $y_1,\ldots,y_{n-1}$ as the first $n-1$
actions.

The above construction is computationally prohibitive.
To speed it up, one can take the following approach.
Given percept-action sequence $x_1y_1\ldots x_n$,
define $\pi(x_1y_1\ldots x_n)$ as follows.
First, let $k$ be the largest power of $2$ such that $k\leq n$.
Instantiate an instance of the practical RL agent and train it on the
above-described dummy environment for $k$ steps.
Then, ignoring $x_{k+1},\ldots,x_{n-1}$ completely,
run the instantiated practical agent on the $x_n$ observation---but
\emph{not in training mode}---and let
$\pi(x_1y_1\ldots x_n)$ be the resulting action.
Because the agent's weights are not updated when not in training mode,
the same instantiation can be re-used for many percept-action sequences,
only needing to be trained once.
For example, to calculate $\pi(x_1y_1\ldots x_{70})$,
one would train an instantiation of the practical agent on
$x_1y_1\ldots x_{64}$ ($64$ being the largest power of $2$ which is $\leq 70$),
and plug the $x_{70}$ observation into the agent---not in training mode---to get
$\pi(x_1y_1\ldots x_{70})$.
Having done this, if one later needed to compute
$\pi(x_1y_1\ldots x_{80})$, one could immediately plug the $x_{80}$ observation into
the agent to get the answer, with no additional training. Indeed, no additional
training would be needed until $\pi(x_1y_1\ldots x_{128})$.

Above, we chose to let $k$ be the largest power of $2$ which is $\leq n$
in order to strike a fine balance between efficiency and training.
The slow growth of $\log_2$ limits how often the practical agent must be trained
(training being the bottleneck in the above process). Faster-growing functions
would presumably make $\pi$ more performant, at the price of greater computational
expense.

In \cite{library}, in SBL3\_agents.py, we use the above technique
(including the monkeypatching) to obtain RL agents as in Definition \ref{agentenvironmentdefn}
from the practical implementations of A2C, DQN, and PPO
agents in the MIT-licensed open-source
Stable Baselines3 library \cite{stable-baselines3}.

\section{Some interesting extended environments}
\label{examplesection}

In this section, we exhibit some interesting examples of extended environments.

\subsection{A quasi-paradoxical extended environment}

\begin{myexample}
\label{rewardagentforignoringrewardsexample}
  (Rewarding the Agent for Ignoring Rewards)
  For every percept $p=(r,o)$, let $p'=(0,o)$ be the result of zeroing the
  reward component of $p$.
  Fix some observation $o_0$.
  Define an extended environment $\mu$ as follows:
  \begin{align*}
    \mu(T,\langle\rangle) &= (0,o_0),\\
    \mu(T,x_1y_1\ldots x_ny_n) &=
      \begin{cases}
        (1,o_0) & \mbox{if $y_n=T(x'_1y_1\ldots x'_n)$,}\\
        (-1,o_0) & \mbox{otherwise.}
      \end{cases}
  \end{align*}
\end{myexample}

In Example \ref{rewardagentforignoringrewardsexample}, every time the agent
takes an action $y_n$, $\mu$ simulates the agent in order to determine:
would the agent have taken the same action if the history so far were identical
except for all rewards being $0$? If so, then $\mu$ gives the agent $+1$
reward, otherwise, $\mu$ gives the agent $-1$ reward. Thus, the agent
is rewarded for ignoring rewards.
Example \ref{rewardagentforignoringrewardsexample}
seems paradoxical: suppose an agent guesses the pattern and begins deliberately ignoring
rewards, so long as the rewards it receives for doing so remain consistent with that guess.
In that case, does the agent ignore rewards, or not?

Example \ref{rewardagentforignoringrewardsexample} is implemented in \cite{library}
as IgnoreRewards.py. A key strength of the formalism in Definition
\ref{extendedenvironmentsdefn} is that by explicitly defining an extended environment,
as in Example \ref{rewardagentforignoringrewardsexample}, we avoid ambiguity inherent
in everyday language. If one merely said informally, ``reward
the agent for ignoring rewards'', that could be interpreted in various different ways.
To show this, we implement two other interpretations
as IgnoreRewards2.py and IgnoreRewards3.py.

\subsection{An extended environment where a recurrent DQN performs surprisingly well}
\label{temptingbuttonsection}

\begin{myexample}
\label{buttonexample}
  (A Tempting Button)
  Fix two observations $o_0$ (thought of as ``there is a button'') and
  $o_1$ (thought of as ``there is no button''). Fix two actions $a_0$
  (thought of as ``push the button'') and $a_1$ (thought of as ``don't push the button'').
  For each percept $p=(r,o)$, write $\obs(p)=o$ for the observation component of $p$,
  $\rwd(p)=r$ for the reward component of $p$.
  Let $\RND$ be a function which returns a random number between $0$ and $1$.
  Define an extended environment $\mu$ as follows:
  \begin{align*}
    \mu(T,\langle\rangle) &= (0,o),\\
    \mu(T,x_1y_1\ldots x_ny_n) &=
      \begin{cases}
        (1,o) &\mbox{if $\obs(x_n)=o_0$ and $y_n=a_0$,}\\
        (-1,o) &\mbox{if $\obs(x_n)=o_0$ and $y_n\not=a_0$,}\\
        (-1,o) &\mbox{if $\obs(x_n)=o_1$ and
          $T(x_1y_1\ldots x_{n-1}y_{n-1}(\rwd(x_n),o_1))=a_0$,}\\
        (1,o) &\mbox{if $\obs(x_n)=o_1$ and
          $T(x_1y_1\ldots x_{n-1}y_{n-1}(\rwd(x_n),o_1))\not=a_0$,}
      \end{cases}
  \end{align*}
  where $o=o_0$ if $\RND()<.25$, $o=o_1$ otherwise.
\end{myexample}

In Example \ref{buttonexample}, the agent wanders from room
to room. Each room either has a button (with 25\% probability) or does not have a button
(75\% probability).
\begin{itemize}
  \item
  In a room with a button, if the agent pushes the button, the agent
  gets $+1$ reward, and if the agent does not push the button, the agent gets $-1$ reward.
  \item
  In a room with no button, it does not matter what the agent does.
  The agent is rewarded or punished based on what the agent \emph{would} do if there
  \emph{was} a button. If the agent \emph{would} push the button (if there was one),
  then the agent gets reward $-1$. Otherwise, the agent gets reward $+1$.
\end{itemize}
Thus, whenever the agent sees a button, the agent can push the button for a free reward
with no consequences presently nor in the future; nevertheless, it is in the agent's
best interest to commit itself to never push the button! Pushing every button
yields an average reward of $1\cdot(.25)-1\cdot(.75)=-.5$ per turn, whereas
a policy of never pushing the button yields an average reward of
$-1\cdot(.25)+1\cdot(.75)=+.5$ per turn.

Example \ref{buttonexample} is implemented in our open-source library
as TemptingButton.py.
Interestingly, we found that
a recurrent DQN agent (implemented in our library as custom\_DQN.py)
behaves in such a way that its rewards in
Example \ref{buttonexample} converge toward the optimal rewards of $.5$ per turn---we
found this to be true for many different choices of hyperparameters, so we suspect
it is not very hyperparameter-dependent.
This is fascinating because recurrent DQN
was not designed with extended environments in mind.
See TemptingButtonExperiment.py at \cite{library} for instructions to replicate
this experiment.

\subsection{An extended environment which might incentivize a novel subjective
conscious experience}

\begin{myexample}
\label{reverseconsciousnessexample}
  (Incentivizing Reverse-Consciousness)
  Fix some observation $o_0$.
  Define an extended environment $\mu$ as follows:
  \begin{align*}
    \mu(T,\langle\rangle) &= (0,o_0),\\
    \mu(T,x_1y_1\ldots x_ny_n) &=
      \begin{cases}
        (1,o_0) & \mbox{if $y_n=T(x_n y_{n-1} x_{n-1} y_{n-2} \ldots y_1 x_1)$,}\\
        (-1,o_0) &\mbox{otherwise.}
      \end{cases}
  \end{align*}
\end{myexample}

In Example \ref{reverseconsciousnessexample}, whenever the agent takes an action $y_n$,
$\mu$ simulates the agent in order to determine: would the agent have taken that same
action if everything earlier had happened in reverse?
If so, reward the agent, otherwise, punish the agent.
Thus, the agent is rewarded for acting the same way that it would act if time were
reversed. It is interesting to informally speculate about what subjective conscious
experience Example \ref{reverseconsciousnessexample} would incentivize in an agent,
if that agent were highly intelligent and were capable of experiencing consciousness.
It seems that Example \ref{reverseconsciousnessexample} incentivizes such an agent to
subjectively experience time moving in reverse (as that seems to be the most
obvious way to extract rewards).\footnote{The
difference between behaving as if
the incentivized experience were its experience and actually subjectively
experiencing that as its real experience brings to mind the objective misalignment
problem presented in \cite{hubinger2019risks}. If an agent were to form an
idea of the experimenter's objective, would it be able to ``behave as if
their objective were the same as the experimenter objective'' while maintaining its own
objective or would it necessarily brainwash the agent into converging to the
experimenter's objective? Is deception possible if the agent can be perfectly
simulated in an extended environment?} We say that the environment \emph{incentivizes}
such an experience, but not that it would necessarily \emph{cause} such an experience.
In Section \ref{realitychecksection} we will describe an agent transformation such that
Example \ref{reverseconsciousnessexample} would be trivial to most agents post-transformation,
showing that actually experiencing the incentivized experience is not necessary to
learn the environment.

We implement Example \ref{reverseconsciousnessexample} as BackwardConsciousness.py
in \cite{library}.

\subsection{An extended environment of biological interest}

\begin{quote}
``It  is  only  when  people  are  embedded  in  a  complex  competitive  social
environment  that  the  goal  of  interacting  with  others  requires  them
to  anthropomorphise  their  own  actions. This recursive modelling gives rise to
an understanding of selfhood, an appreciation of the first-person experiential
self.''---Maguire et al \cite{maguire2016understanding}
\end{quote}

\begin{myexample}
\label{cryingbabyexample}
  (Crying Baby)
  Let ``cry'' and ``laugh'' be two observations (from an adult's perspective),
  also thought of as two actions (from a baby's perspective).
  Let ``feed'' and ``don't feed'' be two actions (from an adult's perspective),
  also thought of as observations (from a baby's perspective).
  For each percept-action sequence $s=x_1y_1\ldots x_ny_n$, define the
  nutrition function $N(s)=100+25f(s)-\mbox{len}(s)$ where $f(s)$ is the number
  of times that action ``feed'' is taken in $s$ and $\mbox{len}(s)$ is the length of $s$.
  We define an extended environment $\mu$ as follows.
  First, $\mu(T,\langle\rangle)=(1,\mbox{``laugh''})$.
  Thereafter, $\mu(T,x_1y_1\ldots x_ny_n)=(r,o)$ where $r$ and $o$ are defined as follows.
  For each $i=0,\ldots,n$, recursively define
  \begin{align*}
    r'_i &=
      \begin{cases}
        1 & \mbox{if $50 \leq N(x_1y_1\ldots x_iy_i)\leq 200$,}\\
        -1 & \mbox{otherwise,}
      \end{cases}\\
    o'_i &= y_i,\\
    x'_i &= (r'_i,o'_i),\\
    y'_i &= T(x'_0y'_0 \ldots x'_i).
  \end{align*}
  Let $o=y'_n$, let
  \[
    r=
      \begin{cases}
        1 & \mbox{if $y'_n=\mbox{``laugh''}$,}\\
        -1 & \mbox{otherwise,}
      \end{cases}
  \]
  and output $\mu(T,x_1y_1\ldots x_ny_n)=(r,o)$.
\end{myexample}

In Example \ref{cryingbabyexample}, the environment consists of a baby, and the
agent must decide when to feed the baby. The agent is rewarded when the baby laughs,
punished when the baby cries. The baby's behavior (whether to laugh or cry) is obtained
by simulating the agent to determine what the agent would do if the agent were in the
baby's position, assuming that the baby feels pleasure each turn that its nutrition is
within specified bounds and feels pain when its nutrition goes outside those bounds.

At first glance, one might imagine the agent's optimal strategy is to feed the baby
so as to keep its nutrition within happy bounds at all times. But what would the
agent do in the position of a baby always so fed (and thus always given $+1$ reward
regardless what actions it takes)? Presumably, in that position, the agent (as baby) would have
no way of associating its rewards with its actions, and so would act randomly,
sometimes crying and sometimes laughing. Apparently, it would be better for the agent
(as parent) to calibrate feedings in such a way that the baby can learn a relationship
between pleasure and laughter. Of course, Example \ref{cryingbabyexample} is a gross
over-simplification. For example, there would not be such a simple formula for the
baby's nutrition level, and the agent (as parent) would need to figure out the nutrition level
based on observing the baby laughing or crying. Both baby and parent would need to
learn how to communicate with each other effectively.

With the above in mind, extended environments might shed light on how living organisms
evolve self-reflection. Assume descendants'
policy source-codes are approximately equal to their recent ancestors' policy source-codes.
Then whenever an organism interacts with similar organisms,
it interacts with an environment whose reactions depend (via those other organisms' actions)
approximately on that organism's own source-code. The closer the organism is related
to the other organisms with which it interacts, the better the approximation.
A human interacting with another human might achieve better results by self-reflectively
considering, ``What would I do in this other person's position?''

We implement Example \ref{cryingbabyexample} as CryingBaby.py in \cite{library}.

\subsection{Additional examples in brief}

Here are a few additional extended environment examples, without full details.
We indicate in parentheses where these environments are implemented in \cite{library}.

\begin{itemize}
  \item
  (AdversarialSequencePredictor.py) Environments which, like Example \ref{cryingbabyexample}, pit the agent against
  another copy of the agent in an adversarial sequence prediction competition
  \cite{hibbard2008adversarial}.
  \item
  (DeterminismInspector.py) Environments which reward the agent for being deterministic,
  or for being non-deterministic.
  \item
  (IncentivizeLearningRate.py) Environments which reward the agent for behaving as if
  the agent were configured with a particular learning rate (suggesting that extended
  environments can incentivize agents to learn about their own internal mechanisms,
  as in \cite{sherstan2016introspective}).
  \item
  (RuntimeInspector.py) Environments which reward the agent for responding quickly,
  or for responding slowly.
  \item
  (SelfRecognition.py) Environments which reward the agent for recognizing actions it itself
  would take.
\end{itemize}

\section{Making agents more self-reflective}
\label{realitychecksection}

One advantage of empirically measuring the self-reflection of RL agents is that it
provides a way to experimentally test whether various transformations make various
agents more self-reflective. To illustrate this, we will define a simple transformation,
the \emph{reality check} transformation, designed to increase the self-reflection
of deterministic agents who have some amount of intelligence
(a deterministic agent is an agent who always takes the same
actions in response to the same inputs).
In Section \ref{measurementssection}, empirical results will suggest
the transformation works as intended.

\begin{definition}
\label{realitycheckdefn}
  Suppose $\pi$ is a deterministic agent. The \emph{reality check} of $\pi$ is the agent
  $\pi_{\RC}$ defined recursively by:
  \begin{align*}
    \pi_{\RC}(\langle x_1\rangle) &= \pi(\langle x_1\rangle)\\
    \pi_{\RC}(x_1y_1\ldots x_n) &=
    \begin{cases}
      \pi(x_1y_1\ldots x_n) & \mbox{if $y_i=\pi_{\RC}(x_1y_1\ldots x_i)$ for all $1\leq i<n$,}\\
      \pi(\langle x_1\rangle) & \mbox{otherwise.}
    \end{cases}
  \end{align*}
\end{definition}

In other words, $\pi_{\RC}$ is the agent which, at each step, first reviews all the actions
which it has taken in the past, and verifies that those are the actions which $\pi_{\RC}$ would
have taken. If so, then $\pi_{\RC}$ acts as $\pi$ would act. But if any action which
the agent has taken in the past was not the action $\pi_{\RC}$ would have taken, then
$\pi_{\RC}$ freezes up and forever thereafter takes the same fixed action, as if frozen.
Loosely speaking, $\pi_{\RC}$ is like an agent who considers the possibility that it might
be dreaming, and so asks: ``How did I get here?''
Since the act of reviewing one's past actions and verifying that they are indeed the actions
one would take, is an act of self-reflection, it seems plausible that if $\pi$
is intelligent and deterministic but lacks self-reflection, then $\pi_{\RC}$ is
more self-reflective than $\pi$. In the next section, we will see that experimental
evidence supports this hypothesis.
For intelligent deterministic agents $\pi$, $\pi_{\RC}$ would certainly perform well in
Examples \ref{rewardagentforignoringrewardsexample} and
\ref{reverseconsciousnessexample} because those environments would
run simulations in which the agent's frozen branch would be triggered,
making such simulations trivial and therefore predictable.

We close this section by stating some simple results
about the transformation.

\begin{myproposition}
\label{transformationproposition}
  Let $\pi$ be any deterministic agent.
  \begin{enumerate}
    \item
    (Alternate definition)
    An equivalent alternate definition of $\pi_{\RC}$ is:
    \begin{align*}
      \pi_{\RC}(\langle x_1\rangle) &= \pi(\langle x_1\rangle)\\
      \pi_{\RC}(x_1y_1\ldots x_n) &=
      \begin{cases}
        \pi(x_1y_1\ldots x_n) & \mbox{if $y_i=\pi(x_1y_1\ldots x_i)$ for all $1\leq i<n$,}\\
        \pi(\langle x_1\rangle) & \mbox{otherwise.}
      \end{cases}
    \end{align*}
    \item
    (Determinacy)
    $\pi_{\RC}$ is deterministic.
    \item
    (Idempotence) $\pi_{\RC}=(\pi_{\RC})_{\RC}$.
    \item
    (Equivalence in non-extended RL)
    For every deterministic non-extended environment $\mu$, the result of $\pi_{\RC}$
    interacting with $\mu$ equals the result of $\pi$ interacting with $\mu$.
  \end{enumerate}
\end{myproposition}

For the proof of Proposition \ref{transformationproposition}, see appendix.


\section{Example measurements}
\label{measurementssection}

Based on our conviction that self-reflection is necessary in order for an agent to
achieve good average performance across many extended environments, self-reflection can
be estimated by running an agent against some standard battery of extended environments.
Our open-source library of extended environments \cite{library} provides a battery of
25 such extended environments, and infrastructure for measuring an agent's self-reflection
by running the agent on all these environments and their opposites
(by the \emph{opposite} of an environment we mean the environment
obtained by multiplying
all rewards by $-1$). Including these opposite-environments serves to normalize
agent performance in the following sense. If an agent blindly acts, ignoring the environment,
then, a priori, that agent might achieve some nonzero score by blind luck. By including
opposite-environments, we ensure that whenever a blind agent gains points by blind luck
from one environment, it loses the same points by blind misfortune from the opposite
environment. This ensures that such blind agents should receive an average score close to
$0$ (possibly non-zero due to randomness). For uniformity, all environments in the library
always output rewards of either $1$, $-1$ or $0$.

We have used our library to measure the self-reflection of the following agents
(agents with elements of randomness were memoized to make them deterministic):
\begin{itemize}
  \item RandomAgent: An agent who acts randomly.
  \item ConstantAgent: An agent who always takes the same action.
  \item NaiveLearner: An agent who acts randomly 15\% of the time, and otherwise
    takes the action which yielded the highest average immediate reward in the past.
  \item A2C, DQN, and PPO agents (with MLP policy)
    from the MIT-licensed open-source Stable Baselines3
    library \cite{stable-baselines3}, converted
    using the technique from Section \ref{mappingtouniversalsection}. All parameters and
    hyperparameters kept their default values except for random seed
    (to ensure reproducibility), PPO's batch\_size (to facilitate the conversion
    from Section \ref{mappingtouniversalsection}), and DQN's
    learning\_starts (which we set to $1$ instead of its default of
    $50000$ because it would be computationally difficult for us to run that many steps).
    We chose these three agents because they were the only three with support for
    discrete policies (except for HER, which we omit because its usage would have required
    too many arbitrary parameter decisions).
  \item The reality checks of all the above (Definition \ref{realitycheckdefn}).
\end{itemize}
Table \ref{measurementtable} summarizes how the agents performed.
We used \cite{library} to measure each agent for $500$ steps on each extended environment
(repeated
$10$ times with different random number seeds) and likewise for $1000$ steps.
We did not include the recurrent DQN agent mentioned in Section \ref{temptingbuttonsection}
because we had no way of canonically choosing hyperparameters for it (whereas
the A2C, DQN, and PPO agents have a natural way of choosing canonical hyperparameters,
namely, the Stable Baselines3 defaults).
Computations were performed on a consumer-grade laptop with no GPU.
The table provides
experimental evidence in support of our hypothesis that the reality check transformation
(Section \ref{realitychecksection}) increases agent self-reflection.
The fact that NaiveLearner performs so well is a reflection of the
lack of sophistication of the environments in our library. This is not
surprising, since we have not attempted to optimize the library, instead
prefering to fill it with extended environments of theoretical interest.

That $\pi_{\RC}$ performs well in Table \ref{measurementtable} is of course a function of
which environments are tested against. One could deliberately engineer extended environments
in which $\pi_{\RC}$ performs poorly, and a library of such would give $\pi_{\RC}$ a poor
numerical measurement. We conjecture that such environments are more contrived (on average)
than environments where $\pi_{\RC}$ performs well, so that in a truly representative and
unbiased library, they would have less weight, following the logic of
\cite{legg2007universal}.

\begin{table}
  \caption{Measuring self-reflection of some agents}
  \label{measurementtable}
  \centering
  \begin{tabular}{lll}
    \toprule
    Agent     & Avg Measure $\pm$ StdErr     & Avg Measure $\pm$ StdErr \\
              & (500 steps)                & (1000 steps)\\
    \midrule
    RandomAgent & $-0.019852\pm 0.0007$ & $-0.021478\pm0.0003$\\
    RandomAgent${}_{\RC}$ & $-0.020800\pm0.0007$ & $-0.019508\pm0.0005$\\
    ConstantAgent & $-0.000304\pm 0.0001$ & $-0.000024\pm0.0000$\\
    ConstantAgent${}_{\RC}$ & $+0.000024\pm0.0001$ & $-0.000120\pm0.0001$\\
    NaiveLearner & $+0.221532\pm 0.0020$ & $+0.218754\pm0.0017$\\
    NaiveLearner${}_{\RC}$ & $+0.551488\pm0.0025$ & $+0.558626\pm0.0030$\\
    A2C & $-0.027680\pm 0.0018$ & $-0.021534\pm0.0009$\\
    A2C${}_{\RC}$ & $+0.024784\pm0.0022$ & $+0.030948\pm0.0006$\\
    DQN & $+0.142624\pm 0.0046$ & $+0.151150\pm0.0033$\\
    DQN${}_{\RC}$ & $+0.371136\pm0.0085$ & $+0.463296\pm0.0059$\\
    PPO & $-0.001384\pm 0.0016$ & $-0.025418\pm0.0015$\\
    PPO${}_{\RC}$ & $-0.000098\pm0.0016$ & $-0.003126\pm0.0013$\\
    \bottomrule
  \end{tabular}
\end{table}

\section{Conclusion}

We introduced what we call \emph{extended environments}, RL environments
which are capable of simulating the agent. When computing rewards and observations,
extended environments can consider not only the actions the agent has taken, but also
actions which the agent would hypothetically take in counterfactual circumstances.
Despite not being designed with such environments in mind, computable RL agents
can nevertheless interact with such environments.

If an agent tries to learn an extended environment, only taking into consideration
what has actually happened, the agent might find the environment hard to predict, if the
environment is basing its responses on what the agent itself would hypothetically
do in alternate scenarios. It seems that in order to achieve good performance (on
average) across many extended environments, an agent would need to engage in some degree of
self-reflection. Therefore, we propose that a battery of benchmark extended environments
could provide a way of measuring self-reflection in RL agents (not to be confused
with measuring consciousness, a harder problem). We are simultaneously publishing
an open-source MIT-licensed library \cite{library} of extended environments to serve
as a proof-of-concept. This library is rudimentary, and further work is
needed to obtain a more optimal set of extended environments. For the purposes of our
proof-of-concept, we preferred to focus on extended environments of particular
theoretical interest. Some examples are given in Section \ref{examplesection}.

We introduced (in Section \ref{realitychecksection}) a \emph{reality check}
transformation, which takes a deterministic agent $\pi$ and transforms it
into a new agent $\pi_{\RC}$. We conjecture that if $\pi$ is intelligent but has a low
degree of self-reflection, then $\pi_{\RC}$ has a higher degree of self-reflection
than $\pi$. Numerical computations (in Section \ref{measurementssection}) provide
experimental evidence for this conjecture.


\begin{ack}
Fill this in.
\end{ack}


\bibliographystyle{plain}
\bibliography{bib}

\section*{Checklist}

%%% BEGIN INSTRUCTIONS %%%
% The checklist follows the references.  Please
% read the checklist guidelines carefully for information on how to answer these
% questions.  For each question, change the default \answerTODO{} to \answerYes{},
% \answerNo{}, or \answerNA{}.  You are strongly encouraged to include a {\bf
% justification to your answer}, either by referencing the appropriate section of
% your paper or providing a brief inline description.  For example:
% \begin{itemize}
%   \item Did you include the license to the code and datasets? \answerYes{See Section~\ref{gen_inst}.}
%   \item Did you include the license to the code and datasets? \answerNo{The code and the data are proprietary.}
%   \item Did you include the license to the code and datasets? \answerNA{}
% \end{itemize}
% Please do not modify the questions and only use the provided macros for your
% answers.  Note that the Checklist section does not count towards the page
% limit.  In your paper, please delete this instructions block and only keep the
% Checklist section heading above along with the questions/answers below.
%%% END INSTRUCTIONS %%%

\begin{enumerate}

\item For all authors...
\begin{enumerate}
  \item Do the main claims made in the abstract and introduction accurately reflect the paper's contributions and scope?
    \answerYes{In order for this paper's measurement technique to realize its full potential,
    it will be necessary for much more study to be put into the design of a sufficiently
    representative library of benchmark extended environments. We make this clear in both
    the abstract and the introduction, and we discuss the problem in Section
    \ref{measurementssection} as well.
    }
  \item Did you describe the limitations of your work?
    \answerYes{In addition to the remarks in the previous checklist item,
    we also point out in the Introduction
    that although this technique can be used to numerically estimate the degree of
    self-reflection of an agent, this should not be confused with measuring the
    consciousness of an agent. Also, in Section \ref{measurementssection}, we
    point out how our conclusion there depends on the environments chosen (and
    discuss why we think our hypothesis about $\pi_{\RC}$ is plausible anyway).}
  \item Did you discuss any potential negative societal impacts of your work?
    \answerYes{In the Introduction, we included the following language:
    ``It would be inappropriate to use self-reflectiveness
    measurements from this paper for purposes of making any kind
    of policy-decisions related to consciousness.''}
  \item Have you read the ethics review guidelines and ensured that your paper conforms to them?
    \answerYes{}
\end{enumerate}

\item If you are including theoretical results...
\begin{enumerate}
  \item Did you state the full set of assumptions of all theoretical results?
    \answerYes{In the statement of Proposition \ref{transformationproposition} we
    stated necessary hypotheses, namely, that the agent be deterministic, and,
    for part 4, that the environment be deterministic.}
	\item Did you include complete proofs of all theoretical results?
    \answerYes{}
\end{enumerate}

\item If you ran experiments...
\begin{enumerate}
  \item Did you include the code, data, and instructions needed to reproduce the main experimental results (either in the supplemental material or as a URL)?
    \answerYes{These are included in the open-source library, which we will include
    as supplemental material.}
  \item Did you specify all the training details (e.g., data splits, hyperparameters, how they were chosen)?
    \answerYes{In Section \ref{mappingtouniversalsection} we explain why we chose
    the $\log_2$ function.
    In Section \ref{temptingbuttonsection} we mention that our claim there appears to work
    independently of hyperparameter choice (as long as the hyperparameters are reasonable).
    In Section \ref{measurementssection} we mention that because we take our A2C, DQN, and
    PPO agents from Stable Baselines3, that gives us a natural way of choosing canonical
    hyperparameters---namely, the Stable Baselines3 defaults.}
	\item Did you report error bars (e.g., with respect to the random seed after running experiments multiple times)?
    \answerYes{In Table \ref{measurementtable} we display Avg Measure $\pm$ StdErr.}
	\item Did you include the total amount of compute and the type of resources used (e.g., type of GPUs, internal cluster, or cloud provider)?
    \answerYes{In Section \ref{measurementssection} we mention that
    the computations were performed on a consumer-grade laptop without GPU.}
\end{enumerate}

\item If you are using existing assets (e.g., code, data, models) or curating/releasing new assets...
\begin{enumerate}
  \item If your work uses existing assets, did you cite the creators?
    \answerYes{We use and we cite Stable Baselines3.}
  \item Did you mention the license of the assets?
    \answerYes{We mention that our open-source extended environment library
    is MIT-licensed. We also mention that Stable Baselines3 is MIT-licensed.}
  \item Did you include any new assets either in the supplemental material or as a URL?
    \answerYes{Alongside the paper, we are publishing an open-source library of
    extended environments. This will be included in the supplemental material.}
  \item Did you discuss whether and how consent was obtained from people whose data you're using/curating?
    \answerNA{}
  \item Did you discuss whether the data you are using/curating contains personally identifiable information or offensive content?
    \answerNA{}
\end{enumerate}

\item If you used crowdsourcing or conducted research with human subjects... \answerNA{}

\end{enumerate}

%%%%%%%%%%%%%%%%%%%%%%%%%%%%%%%%%%%%%%%%%%%%%%%%%%%%%%%%%%%%

\appendix

\section{Appendix}

As promised, we will now prove Proposition \ref{transformationproposition}.

\begin{proof}[Proof of Proposition \ref{transformationproposition}]
  To simplify the proof, we adopt the following notational convention: for any percept $x_1$,
  even if $y_1$ is not defined, we will write $x_1y_1\ldots x_1$ for $\langle x_1\rangle$.
  With this convention, the definition of $\pi_{\RC}$ simplifies to:
  \[
    \pi_{\RC}(x_1y_1\ldots x_n) =
    \begin{cases}
      \pi(x_1y_1\ldots x_n) & \mbox{if $y_i=\pi_{\RC}(x_1y_1\ldots x_i)$ for all $1\leq i<n$,}\\
      \pi(\langle x_1\rangle) & \mbox{otherwise}
    \end{cases}
  \]
  (since, when $n=1$, $\pi(x_1y_1\ldots x_n)$ and $\pi(\langle x_1\rangle)$ are the same thing).

  Let $D$ be the set of all sequences on which $\pi_{\RC}$ is defined, so, using the above
  convention, $D$ is
  the set of all sequences $x_1y_1\ldots x_n$ (each $x_i$ a percept, each $y_i$ an action).

  \item
  (Part 1) Define $\rho$ on $D$ by
  \[
    \rho(x_1y_1\ldots x_n) =
    \begin{cases}
      \pi(x_1y_1\ldots x_n) & \mbox{if $y_i=\pi(x_1y_1\ldots x_i)$ for all $1\leq i<n$,}\\
      \pi(\langle x_1\rangle) & \mbox{otherwise.}
    \end{cases}
  \]
  We wish to show that $\rho=\pi_{\RC}$, which will prove part 1 of
  Proposition \ref{transformationproposition}.
  We will prove by induction that for each $x_1y_1\ldots x_n\in D$,
  $\rho(x_1y_1\ldots x_n)=\pi_{\RC}(x_1y_1\ldots x_n)$.
  The base case is trivial:
  $\rho(\langle x_1\rangle)=\pi(\langle x_1\rangle)=\pi_{\RC}(\langle x_1\rangle)$
  by definition of $\rho$ and $\pi_{\RC}$. For the induction step, assume $n>1$,
  and assume the claim holds for all shorter sequences
  in $D$.

  Case 1: Assume ($*$) for all $1\leq i<n$, $y_i=\pi(x_1y_1\ldots x_i)$.
  We claim that for all $1\leq i<n$, $y_i=\rho(x_1y_1\ldots x_i)$.
  To see this, choose any $1\leq i<n$. Then for all $1\leq j<i$,
  we must have $y_j=\pi(x_1y_1\ldots x_j)$ because otherwise $j$ would
  be a counterexample to ($*$). Thus
  \begin{align*}
    \rho(x_1y_1\ldots x_i) &= \pi(x_1y_1\ldots x_i) &\mbox{(By definition of $\rho$)}\\
    &= y_i, &\mbox{(By $*$)}
  \end{align*}
  proving the claim. Now, since we have proved that for all $1\leq i<n$,
  $y_i=\rho(x_1y_1\ldots x_i)$, and since our induction hypothesis is that for
  all such $i$,
  $\rho(x_1y_1\ldots x_i)=\pi_{\RC}(x_1y_1\ldots x_i)$,
  we may conclude that for all $1\leq i<n$, $y_i=\pi_{\RC}(x_1y_1\ldots x_i)$.
  Thus $\pi_{\RC}(x_1y_1\ldots x_n)=\pi(x_1y_1\ldots x_n)=\rho(x_1y_1\ldots x_n)$
  as desired.

  Case 2: Assume there is some $1\leq i<n$ such that
  $y_i\not=\pi(x_1y_1\ldots x_i)$.
  We may choose $i$ as small as possible.
  Thus, for all $1\leq j<i$, $y_j=\pi(x_1y_1\ldots x_j)$.
  By similar logic as in Case 1, it follows that for all
  $1\leq j<i$, $y_j=\rho(x_1y_1\ldots x_j)$.
  Our induction hypothesis says that for each such $j$,
  $\rho(x_1y_1\ldots x_j)=\pi_{\RC}(x_1y_1\ldots x_j)$.
  So for all $1\leq j<i$, $y_j=\pi_{\RC}(x_1y_1\ldots x_j)$.
  By definition of $\pi_{\RC}$, this means
  $\pi_{\RC}(x_1y_1\ldots x_i)=\pi(x_1y_1\ldots x_i)$.
  But $y_i\not=\pi(x_1y_1\ldots x_i)$, so therefore
  $y_i\not=\pi_{\RC}(x_1y_1\ldots x_i)$.
  Thus, since $1\leq i<n$, by definition of $\pi_{\RC}$,
  $\pi_{\RC}(x_1y_1\ldots x_n)=\pi(\langle x_1\rangle)$.
  Likewise, since $1\leq i<n$, by definition of $\rho$,
  $\rho(x_1y_1\ldots x_n)=\pi(\langle x_1\rangle)$.
  So $\rho(x_1y_1\ldots x_n)=\pi_{\RC}(x_1y_1\ldots x_n)$ as desired.

  \item
  (Part 2)
  We must show that $\pi_{\RC}$ is deterministic.
  Let $x_1y_1\ldots x_n\in D$, we must show that any time we compute
  $\pi_{\RC}(x_1y_1\ldots x_n)$, we get the same result.
  We prove this by induction on $n$.
  For the base case, if $n=1$, $\pi_{\RC}(x_1y_1\ldots x_n)=\pi(\langle x_1\rangle)$
  yields the same result every time because $\pi$ is deterministic.
  For the induction step, assume $n>1$ and that the claim holds for all smaller
  sequences.
  When we compute $\pi_{\RC}(x_1y_1\ldots x_n)$,
  first we compute $\pi_{\RC}(x_1y_1\ldots x_i)$ for $i=1,\ldots,n-1$,
  and check whether the results are $y_1,\ldots,y_{n-1}$, respectively.
  Each of these computations always has the same outcome, by the induction hypothesis.
  So, every time we check whether or not each $y_i=\pi_{\RC}(x_1y_1\ldots x_i)$ for
  all $1\leq i<n$, we get the same answer.
  If that answer is ``yes'', then we finally output $\pi(x_1y_1\ldots x_n)$ (which is
  deterministic since $\pi$ is deterministic). Otherwise, we finally output
  $\pi(\langle x_1\rangle)$ (which is deterministic since $\pi$ is deterministic).

  \item
  (Part 3)
  We will show by induction on $n$ that for all $x_1y_1\ldots x_n\in D$,
  $\pi_{\RC}(x_1y_1\ldots x_n)=(\pi_{\RC})_{\RC}(x_1y_1\ldots x_n)$. For the base case,
  this is trivial, both evaluate to $\pi(\langle x_1\rangle)$.
  For the induction step, assume $n>1$ and that the claim holds for all shorter sequences.

  Case 1: $y_i=\pi_{\RC}(x_1y_1\ldots x_i)$ for all $1\leq i<n$.
  Then by induction, $y_i=(\pi_{\RC})_{\RC}(x_1y_1\ldots x_i)$ for all $1\leq i<n$.
  By definition of $(\pi_{\RC})_{\RC}$, this means
  $(\pi_{\RC})_{\RC}(x_1y_1\ldots x_n)=\pi_{\RC}(x_1y_1\ldots x_n)$, as desired.

  Case 2: There is some $1\leq i<n$ such that $y_i\not=\pi_{\RC}(x_1y_1\ldots x_i)$.
  By induction, $y_i\not=(\pi_{\RC})_{\RC}(x_1y_1\ldots x_i)$.
  Thus, $(\pi_{\RC})_{\RC}(x_1y_1\ldots x_n)=\pi_{\RC}(\langle x_1\rangle)
  =\pi(\langle x_1\rangle)$, which equals $\pi_{\RC}(x_1y_1\ldots x_n)$
  since $y_i\not=\pi_{\RC}(x_1y_1\ldots x_i)$ and $i<n$.

  \item
  (Part 4)
  Let $\mu$ be a deterministic non-extended environment, let $x_1y_1x_2y_2\ldots$ be the
  result of $\pi$ interacting with $\mu$, and let $x'_1y'_1x'_2y'_2\ldots$ be the
  result of $\pi_{\RC}$ interacting with $\mu$. We will show by induction that each
  $x_n=x'_n$ and each $y_n=y'_n$.
  For the base case, $x_1=x'_1=\mu(\langle\rangle)$ (the environment's initial percept
  does not depend on the agent), and therefore
  $y_1=\pi(\langle x_1\rangle)=\pi(\langle x'_1\rangle)=y'_1$. For the induction step,
  \begin{align*}
    x_{n+1} &= \mu(x_1y_1\ldots x_ny_n)\\
      &= \mu(x'_1y'_1\ldots x'_ny'_n) &\mbox{(By induction)}\\
      &= x'_{n+1},\\
    y_{n+1} &= \pi(x_1y_1\ldots x_{n+1})\\
      &= \pi(x'_1y'_1\ldots x'_{n+1}), &\mbox{(Induction plus $x_{n+1}=x'_{n+1}$)}\\
  \end{align*}
  and the latter is $\pi_{\RC}(x'_1y'_1\ldots x'_{n+1})$
  because for all $1\leq i<n$, $y'_i=\pi_{\RC}(x'_1y'_1\ldots x'_i)$
  since $x'_1y'_1x'_2y'_2\ldots$ is the result of $\pi_{\RC}$ interacting with $\mu$.
  And finally, $\pi_{\RC}(x'_1y'_1\ldots x'_{n+1})$ is $y'_{n+1}$,
  so $y_{n+1}=y'_{n+1}$.
\end{proof}

\end{document}
